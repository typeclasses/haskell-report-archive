%
% $Header$
%
% The paragraph describing the formats of standard representations might
% be deleted, since the info is already in the Prelude.  
% Note that there is a difference in the way readsPrec and showsPrec are defined.
% showsPrec is exact Haskell text, readsPrec uses an auxiliary function which
% isn't quite Haskell.  

\section{Specification of Derived Instances}
\label{derived-appendix}

If \mbox{$\it T$} is an algebraic datatype declared by:\index{algebraic datatype}
\[\begin{array}{lcl}
\mbox{$\it \makebox{\tt data\ }c\makebox{\tt \ =>}\ T\ u_1\ \ldots \ u_k$}&\mbox{\tt =}&\mbox{$\it K_1\ t_{11}\ \ldots \ t_{1k_1}\ \makebox{\tt |}\ \cdots\ \makebox{\tt |}\ K_n\ t_{n1}\ \ldots \ t_{nk_n}$}\\
& & \mbox{$\it \makebox{\tt deriving\ (}C_1\makebox{\tt ,}\ \ldots \makebox{\tt ,}\ C_m\makebox{\tt )}$}
\end{array}\]
(where \mbox{$\it m\geq0$} and the parentheses may be omitted if \mbox{$\it m=1$}) then
{\em a derived instance declaration is possible} for a class \mbox{$\it C$} 
if and only if these conditions hold:
\begin{enumerate}
\item
\mbox{$\it C$} is one of \mbox{\tt Eq}, \mbox{\tt Ord}, \mbox{\tt Enum}, \mbox{\tt Ix}, \mbox{\tt Text}, or \mbox{\tt Binary}.

\item
There is a context \mbox{$\it c'$} such that \mbox{$\it c'\ \Rightarrow\ C\ t_{ij}$}
holds for each of the constituent types \mbox{$\it t_{ij}$}.

\item
If \mbox{$\it C$} is either \mbox{\tt Ix} or \mbox{\tt Enum}, then further constraints must be
satisfied as described under the paragraphs for \mbox{\tt Ix} and \mbox{\tt Enum}
later in this section.

\item
There must be no explicit instance declaration elsewhere in the module which
makes \mbox{$\it T\ u_1\ \ldots \ u_k$} an instance of \mbox{$\it C$}.
% or of any of \mbox{$\it C$}'s superclasses.
\end{enumerate}

If the \mbox{\tt deriving} form is present (as in the above 
general \mbox{\tt data} declaration),
an instance declaration is automatically generated for \mbox{$\it T\ u_1\ \ldots \ u_k$}
over each class \mbox{$\it C_i$} and each of \mbox{$\it C_i$}'s superclasses.
If the derived instance declaration is impossible for any of the \mbox{$\it C_i$}
then a static error results.
If no derived instances are required, the \mbox{\tt deriving} form may be
omitted or the form \mbox{\tt deriving\ ()} may be used.

% OLD:
%If the \mbox{\tt deriving} form is omitted then instance
%declarations are derived for the datatype in as many of the six
%classes mentioned above as is possible; that is, no static error can occur.
%Since datatypes may be recursive, the implied inclusion in
%these classes may also be recursive, and the largest
%possible set of derived instances is generated.  For example,
%\bprog
%@%@
%data  T1 a = C1 (T2 a) | Nil1
%data  T2 a = C2 (T1 a) | Nil2
%@%@
%\eprog
%Because the \mbox{\tt deriving} form is omitted, we would expect derived
%instances for \mbox{\tt Eq} (for example).  But \mbox{\tt T1} is in \mbox{\tt Eq} only if \mbox{\tt T2}
%is, and \mbox{\tt T2} is in \mbox{\tt Eq} only if \mbox{\tt T1} is.  The largest solution has
%both types in \mbox{\tt Eq}, and thus both derived instances are generated.

Each derived instance declaration will have the form:
\[
\mbox{$\it \makebox{\tt instance\ (}c\makebox{\tt ,}\ C'_1\ u'_1\makebox{\tt ,}\ \ldots \makebox{\tt ,}\ C'_j\ u'_j\ \makebox{\tt )\ =>}\ C_i\ (T\ u_1\ \ldots \ u_k)\ \makebox{\tt where}\ \makebox{\tt {\char'173}}\ d\ \makebox{\tt {\char'175}}$}
\]
where \mbox{$\it d$} is derived automatically depending on \mbox{$\it C_i$} and the data
type declaration for \mbox{$\it T$} (as will be described in the remainder of
this section), and \mbox{$\it u'_1$} through \mbox{$\it u'_j$} form a subset of \mbox{$\it u_1$}
through \mbox{$\it u_k$}.
%% Yale nuked this:
%% The class assertions \mbox{$\it C'\ u'$} are constraints on \mbox{$\it T$}'s
%% type variables that are inferred from the instance declarations of the
%% constituent types \mbox{$\it t_{ij}$}.  For example, consider:
%% \bprog
%% @
%% data  T1 a = C1 (T2 a) deriving Eq
%% data  T2 a = C2 a      deriving ()
%% @
%% \eprog
%% And consider these three different instances for \mbox{\tt T2} in \mbox{\tt Eq}:\nopagebreak
%% \bprog
%% @
%% instance            Eq (T2 a) where (C2 x) == (C2 y)  =  True
%% 
%% instance (Eq  a) => Eq (T2 a) where (C2 x) == (C2 y)  =  x == y
%% 
%% instance (Ord a) => Eq (T2 a) where (C2 x) == (C2 y)  =  x > y
%% @
%% \eprog
%% The corresponding derived instances for \mbox{\tt T1} in \mbox{\tt Eq} are:
%% \bprog
%% @
%% instance            Eq (T1 a) where (C1 x) == (C1 y)  =  x == y
%% 
%% instance (Eq  a) => Eq (T1 a) where (C1 x) == (C1 y)  =  x == y
%% 
%% instance (Ord a) => Eq (T1 a) where (C1 x) == (C1 y)  =  x == y
%% @
%% \eprog
When inferring the context for the derived instances, type synonyms
must be expanded out first.\index{type synonym}
The free variables of the declarations $d$ are all functions
defined in the standard prelude.
%These prelude functions must 
%be in scope whenever \mbox{\tt deriving} instances are used that
%mention them.
The remaining details of the derived
instances for each of the six classes are now given.

\paragraph*{Derived instances of \mbox{\tt Eq} and \mbox{\tt Ord}.}
\index{Eq@{\ptt Eq}!derived instance}
\index{Ord@{\ptt Ord}!derived instance}
The operations automatically introduced by derived instances
of \mbox{\tt Eq} and \mbox{\tt Ord} are \mbox{\tt (==)}\indextt{==}, \mbox{\tt (/=)}\indextt{/=},
\mbox{\tt (<)}\indextt{<}, \mbox{\tt (<=)}\indextt{<=}, \mbox{\tt (>)}\indextt{>},
\mbox{\tt (>=)}\indextt{>=}, \mbox{\tt max}\indextt{max}, and 
\mbox{\tt min}\indextt{min}.  The latter six operators are defined so
as to compare their arguments lexicographically with respect to
the constructor set given, with earlier constructors in the datatype
declaration counting as smaller than later ones.  For example, for the
\mbox{\tt Bool} datatype, we have that \mbox{\tt True\ >\ False\ ==\ True}.

\paragraph*{Derived instances of \mbox{\tt Ix}.}
\index{Ix@{\ptt Ix}!derived instance}
The derived instance declarations for the class \mbox{\tt Ix}
introduce the overloaded functions
\mbox{\tt range}\indextt{range}, \mbox{\tt index}\indextt{index}, and
\mbox{\tt inRange}\indextt{inRange}.  The operation \mbox{\tt range} takes a (lower,
upper) bound pair, and returns a list of all indices in this range, in
ascending order.  The operation \mbox{\tt inRange} is a predicate taking a
(lower, upper) bound pair and an index and returning \mbox{\tt True} if the
index is contained within the specified range.  The operation \mbox{\tt index}
takes a (lower, upper) bound pair and an index and returns an integer,
the position of the index within the range.

Derived instance declarations for the class \mbox{\tt Ix} are only possible
for enumerations\index{enumeration} (i.e.~datatypes having
only nullary constructors) and single-constructor datatypes
(including tuples) whose constituent types are instances of \mbox{\tt Ix}.  
\begin{itemize}
\item
For an {\em enumeration}, the nullary constructors are assumed to be
numbered left-to-right with the indices 0 through $n-1\/$.  For example,
given the datatype:
\bprog
\mbox{\tt data\ \ Colour\ =\ Red\ |\ Orange\ |\ Yellow\ |\ Green\ |\ Blue\ |\ Indigo\ |\ Violet}
\eprog
we would have:
\bprog
\mbox{\tt range\ \ \ (Yellow,Blue)\ \ \ \ \ \ \ \ ==\ \ [Yellow,Green,Blue]}\\
\mbox{\tt index\ \ \ (Yellow,Blue)\ Green\ \ ==\ \ 1}\\
\mbox{\tt inRange\ (Yellow,Blue)\ Red\ \ \ \ ==\ \ False}
\eprog
\item
For {\em single-constructor datatypes}, the derived instance declarations
are created as shown for tuples in
Figure~\ref{prelude-index}.
\end{itemize}

%Instances of the class \mbox{\tt Ix}\indextt{Ix} are typically used as
%indices of arrays; a one-dimensional array might have index type
%\mbox{\tt Int}, a two-dimensional array, \mbox{\tt (Int,Char)}, and so forth.  (See
%Section~\ref{arrays}.)

\begin{figure}
\outline{
\mbox{\tt class\ \ (Ord\ a)\ =>\ Ix\ a\ where}\\
\mbox{\tt \ \ \ \ \ \ \ \ range\ \ \ \ \ \ \ \ \ \ \ ::\ (a,a)\ ->\ [a]}\\
\mbox{\tt \ \ \ \ \ \ \ \ index\ \ \ \ \ \ \ \ \ \ \ ::\ (a,a)\ ->\ a\ ->\ Int}\\
\mbox{\tt \ \ \ \ \ \ \ \ inRange\ \ \ \ \ \ \ \ \ ::\ (a,a)\ ->\ a\ ->\ Bool}\\
\mbox{\tt }\\[-8pt]
\mbox{\tt rangeSize\ \ \ \ \ \ \ \ \ \ \ \ \ \ \ ::\ (Ix\ a)\ =>\ (a,a)\ ->\ Int}\\
\mbox{\tt rangeSize\ (l,u)\ \ \ \ \ \ \ \ \ =\ \ index\ (l,u)\ u\ +\ 1}\\
\mbox{\tt }\\[-8pt]
\mbox{\tt instance\ \ (Ix\ a,\ Ix\ b)\ \ =>\ Ix\ (a,b)\ where}\\
\mbox{\tt \ \ \ \ \ \ \ \ range\ ((l,l'),(u,u'))}\\
\mbox{\tt \ \ \ \ \ \ \ \ \ \ \ \ \ \ \ \ =\ [(i,i')\ |\ i\ <-\ range\ (l,u),\ i'\ <-\ range\ (l',u')]}\\
\mbox{\tt \ \ \ \ \ \ \ \ index\ ((l,l'),(u,u'))\ (i,i')}\\
\mbox{\tt \ \ \ \ \ \ \ \ \ \ \ \ \ \ \ \ =\ \ index\ (l,u)\ i\ *\ rangeSize\ (l',u')\ +\ index\ (l',u')\ i'}\\
\mbox{\tt \ \ \ \ \ \ \ \ inRange\ ((l,l'),(u,u'))\ (i,i')}\\
\mbox{\tt \ \ \ \ \ \ \ \ \ \ \ \ \ \ \ \ =\ inRange\ (l,u)\ i\ {\char'46}{\char'46}\ inRange\ (l',u')\ i'}\\
\mbox{\tt }\\[-8pt]
\mbox{\tt --\ Instances\ for\ other\ tuples\ are\ obtained\ from\ this\ scheme:}\\
\mbox{\tt --}\\
\mbox{\tt --\ \ instance\ \ (Ix\ a1,\ Ix\ a2,\ ...\ ,\ Ix\ ak)\ =>\ Ix\ (a1,a2,...,ak)\ \ where}\\
\mbox{\tt --\ \ \ \ \ \ range\ ((l1,l2,...,lk),(u1,u2,...,uk))\ =}\\
\mbox{\tt --\ \ \ \ \ \ \ \ \ \ [(i1,i2,...,ik)\ |\ i1\ <-\ range\ (l1,u1),}\\
\mbox{\tt --\ \ \ \ \ \ \ \ \ \ \ \ \ \ \ \ \ \ \ \ \ \ \ \ \ \ \ \ i2\ <-\ range\ (l2,u2),}\\
\mbox{\tt --\ \ \ \ \ \ \ \ \ \ \ \ \ \ \ \ \ \ \ \ \ \ \ \ \ \ \ \ ...}\\
\mbox{\tt --\ \ \ \ \ \ \ \ \ \ \ \ \ \ \ \ \ \ \ \ \ \ \ \ \ \ \ \ ik\ <-\ range\ (lk,uk)]}\\
\mbox{\tt --}\\
\mbox{\tt --\ \ \ \ \ \ index\ ((l1,l2,...,lk),(u1,u2,...,uk))\ (i1,i2,...,ik)\ =}\\
\mbox{\tt --\ \ \ \ \ \ \ \ index\ (lk,uk)\ ik\ +\ rangeSize\ (lk,uk)\ *\ (}\\
\mbox{\tt --\ \ \ \ \ \ \ \ \ index\ (lk-1,uk-1)\ ik-1\ +\ rangeSize\ (lk-1,uk-1)\ *\ (}\\
\mbox{\tt --\ \ \ \ \ \ \ \ \ \ ...}\\
\mbox{\tt --\ \ \ \ \ \ \ \ \ \ \ index\ (l1,u1)))}\\
\mbox{\tt --}\\
\mbox{\tt --\ \ \ \ \ \ inRange\ ((l1,l2,...lk),(u1,u2,...,uk))\ (i1,i2,...,ik)\ =}\\
\mbox{\tt --\ \ \ \ \ \ \ \ \ \ inRange\ (l1,u1)\ i1\ {\char'46}{\char'46}\ inRange\ (l2,u2)\ i2\ {\char'46}{\char'46}}\\
\mbox{\tt --\ \ \ \ \ \ \ \ \ \ \ \ \ \ ...\ {\char'46}{\char'46}\ inRange\ (lk,uk)\ ik}
}
\ecaption{Index classes and instances}
\label{prelude-index}
\indextt{Ix}                                                
\indextt{range}\indextt{index}\indextt{inRange}   
\indextt{rangeSize}                                         
\end{figure}

\paragraph*{Derived instances of \mbox{\tt Enum}.}
\index{Enum@{\ptt Enum}!derived instance}
Derived instance declarations for the class \mbox{\tt Enum} are only
possible for enumerations, using the same ordering assumptions made
for \mbox{\tt Ix}.  They introduce the operations
\mbox{\tt enumFrom}\indextt{enumFrom},
\mbox{\tt enumFromThen}\indextt{enumFromThen}, \mbox{\tt enumFromTo}\indextt{enumFromTo}, and
\mbox{\tt enumFromThenTo}\indextt{enumFromThenTo},
which are used to define arithmetic sequences as described
in Section~\ref{arithmetic-sequences}.

\mbox{\tt enumFrom\ n} returns a list corresponding to the complete enumeration
of \mbox{\tt n}'s type starting at the value \mbox{\tt n}.
Similarly, \mbox{\tt enumFromThen\ n\ n'} is the enumeration starting at \mbox{\tt n}, but
with second element \mbox{\tt n'}, and with subsequent elements generated at a
spacing equal to the difference between \mbox{\tt n} and \mbox{\tt n'}.
\mbox{\tt enumFromTo} and \mbox{\tt enumFromThenTo} are as defined by the
default methods
\index{default method}
for \mbox{\tt Enum} (see Figure~\ref{standard-classes},
page~\pageref{standard-classes}).

\paragraph*{Derived instances of \mbox{\tt Binary}.}
\index{Binary@{\ptt Binary}!derived instance}
The \mbox{\tt Binary} class is used primarily for transparent I/O (see
Section~\ref{io-modes}).  The operations automatically introduced
by derived instances of \mbox{\tt Binary} are \mbox{\tt readBin}\indextt{readBin} and
\mbox{\tt showBin}\indextt{showBin}.  They coerce values to and
from the primitive abstract type \mbox{\tt Bin} (see Section~\ref{bin-type}).
An implementation must be able to create derived instances of \mbox{\tt Binary}
for any type \mbox{$\it t$} not containing a function type.

\mbox{\tt showBin} is analogous to \mbox{\tt shows}, taking two arguments: the first
is the value to be coerced, and the second is a \mbox{\tt Bin} value to which
the result is to be concatenated.  \mbox{\tt readBin} is analogous to \mbox{\tt reads},
``parsing'' its argument and returning a pair consisting of the
coerced value and any remaining \mbox{\tt Bin} value.  

Derived versions of \mbox{\tt showBin} and \mbox{\tt readBin} must obey this
property:
\[
\mbox{$\it \makebox{\tt readBin\ (showBin\ }v\ b\makebox{\tt )\ ==\ (}v\makebox{\tt ,}b\makebox{\tt )}$}
\]
for any \mbox{\tt Bin} value \mbox{$\it b$} and value \mbox{$\it v$} whose type is an instance of the
class \mbox{\tt Binary}.

\paragraph*{Derived instances of \mbox{\tt Text}.}
\index{Text@{\ptt Text}!derived instance}
The operations automatically introduced by derived instances
of \mbox{\tt Text} are \mbox{\tt showsPrec}\indextt{showsPrec}, \mbox{\tt readsPrec}\indextt{readsPrec},
\mbox{\tt showList}\indextt{showList} and \mbox{\tt readList}\indextt{readList}.
They are used to coerce values into strings and parse strings into
values.

The function \mbox{\tt showsPrec\ d\ x\ r} accepts a precedence level \mbox{\tt d}
(a number from \mbox{\tt 0} to \mbox{\tt 10}), a value \mbox{\tt x}, and a string \mbox{\tt r}.
It returns a string representing \mbox{\tt x} concatenated to \mbox{\tt r}.
\mbox{\tt showsPrec} satisfies the law:
\[
\mbox{$\it \makebox{\tt showsPrec\ d\ x\ r\ ++\ s\ \ ==\ \ showsPrec\ d\ x\ (r\ ++\ s)}$}
\]
The representation will be enclosed in parentheses if the precedence
of the top-level constructor operator in \mbox{\tt x} is less than \mbox{\tt d}.  Thus,
if \mbox{\tt d} is \mbox{\tt 0} then the result is never surrounded in parentheses; if
\mbox{\tt d} is \mbox{\tt 10} it is always surrounded in parentheses, unless it is an
atomic expression.  The extra parameter \mbox{\tt r} is essential if tree-like
structures are to be printed in linear time rather than time quadratic
in the size of the tree.

The function \mbox{\tt readsPrec\ d\ s} accepts a precedence level \mbox{\tt d} (a number
from \mbox{\tt 0} to \mbox{\tt 10}) and a string \mbox{\tt s}, and returns a list of pairs
\mbox{\tt (x,r)} such that \mbox{\tt showsPrec\ d\ x\ r\ ==\ s}.  \mbox{\tt readsPrec} is a parse
function, returning a list of (parsed value, remaining string) pairs.
If there is no successful parse, the returned list is empty.

\mbox{\tt showList} and \mbox{\tt readList} allow lists of objects to be represented
using non-standard denotations.  This is especially useful for strings
(lists of \mbox{\tt Char}).

%Because a string is a list of characters, \mbox{\tt showsPrec} of a string
%such as \mbox{\tt "abc"} will result in the string 
%\mbox{\tt "[}\fwq\mbox{\tt a}\fwq \mbox{\tt ,} \fwq\mbox{\tt b}\fwq \mbox{\tt ,} \fwq\mbox{\tt c}\fwq \mbox{\tt ]"}.  Because
%\mbox{\tt "{\char'134}"abc{\char'134}""} would be a better representation,
%the operators \mbox{\tt showList}
%and \mbox{\tt readList} are provided in the class \mbox{\tt Text} for coercing {\em
%lists} of values to and from strings.  In particular, \mbox{\tt showsPrec} of a
%string will yield the verbose form above, and \mbox{\tt showList} will yield
%the compact version.  For most other datatypes, \mbox{\tt showList} and
%\mbox{\tt readList} will yield the same result as \mbox{\tt showsPrec} and \mbox{\tt readsPrec}.

For convenience, the standard prelude provides the following auxiliary functions:
\bprog
\mbox{\tt shows\ \ \ \ =\ \ showsPrec\ 0}\\
\mbox{\tt reads\ \ \ \ =\ \ readsPrec\ 0}\\
\mbox{\tt show\ x\ \ \ =\ \ shows\ x\ ""}\\
\mbox{\tt read\ s\ \ \ =\ \ x\ \ where\ [(x,"")]\ =\ reads\ s}
\eprog
\mbox{\tt shows} and \mbox{\tt reads} use a default precedence of 0, and \mbox{\tt show} and \mbox{\tt read}
assume that the result is not being appended to an initial string.

The instances of \mbox{\tt Text} for the standard types \mbox{\tt Int}, \mbox{\tt Integer}, \mbox{\tt Float},
\mbox{\tt Double}, \mbox{\tt Char}, lists, tuples, and rational and complex numbers are defined in the 
standard prelude (see Appendix~\ref{stdprelude}).
For characters and strings, the control characters that have special
representations (\mbox{\tt {\char'134}n} etc.) are shown as such by \mbox{\tt showsPrec};
otherwise, ASCII mnemonics are used.
Non-ASCII characters are shown by decimal escapes.
Floating point numbers are represented by decimal numbers
of sufficient precision to guarantee \mbox{\tt read\ .\ show} is an identity
function.  If $b$ is the floating-point radix and there are
$w$ base-$b$ digits in the floating-point significand,
the number of decimal digits required is
$d = \lceil w \log_{10} b \rceil + 1$ \cite{matula70}.
Numbers are shown in non-exponential format if this requires
only $d$ digits; otherwise, they are shown in exponential format,
with one digit before the decimal point.  \mbox{\tt readsPrec} allows
an exponent to be unsigned or signed with \mbox{\tt +} or \mbox{\tt -}; \mbox{\tt showsPrec}
shows a positive exponent without a sign.

\mbox{\tt readsPrec} will parse any valid representation of the standard types 
apart from lists, for
which only the bracketed form \mbox{\tt [}\ldots\mbox{\tt ]} is accepted. See
Appendix~\ref{stdprelude} for full details.

%Figure~\ref{derived-text} gives the general form of a derived instance
%of \mbox{\tt Text} for a user-defined datatype:
%\[
%\mbox{$\it \makebox{\tt data}\ c\ \makebox{\tt =>}\ T\ u_1\ \ldots \ u_k\ \makebox{\tt =}\ \ldots $}
%\]
%\mbox{\tt showsPrec} and \mbox{\tt readsPrec} are as
%in Appendices~\ref{showsPrec-spec} and \ref{readsPrec-spec}.  The omitted
%definitions of \mbox{\tt readList} and \mbox{\tt showList} in
%Figure~\ref{standard-classes} (page~\pageref{standard-classes})
%are:
%\bprog
%@
%readList:: ReadS [a]
%readList r = [pr | (\mbox{$\it [$},s) <- lex r,
%                 pr      <- readl s    ]
%           where readl s = [([],t) | (\mbox{$\it ]$},t) <- lex s] ++
%                          [(x:xs,v) |  (x,t) <- reads s,
%                                       (\mbox{$\it ,$},u) <- lex t,
%                                       (xs,v) <- readl u       ]
%
%showList:: [a] -> ShowS
%showList xs = showChar '[' . showl xs
%            where
%            showl [] = showChar ']'
%            showl (x:xs) = shows x . showChar ',' . showl xs
%@
%\eprog
%\begin{figure}
%\outline{
%@
%instance (C, Text u1, ..., Text uk) => Text (T u1 ... uk) where
%       showsPrec = ...
%       readsPrec = ...
%@
%}
%\caption{General form of a derived instance of \mbox{\tt Text}}
%\label{derived-text}
%\end{figure}

\subsection{Specification of \mbox{\tt showsPrec}}
\label{showsPrec-spec}
\indextt{showsPrec}

As described in Section~\ref{derived-decls}, \mbox{\tt showsPrec} has the type
\[
\mbox{$\it \makebox{\tt (Text\ a)\ =>\ Int\ ->\ a\ ->\ String\ ->\ String}$}
\]
The first parameter is a
precedence in the range 0 to 10, the second is the value to be
converted into a string, and the third is the string
to append to the end of the result.

For all constructors \mbox{\tt Con} defined by some \mbox{\tt data} declaration
such as:
\[
\mbox{$\it \makebox{\tt data}\ c\ \makebox{\tt =>}\ T\ u_1\ \ldots \ u_k\ \makebox{\tt =}\ \ldots \ \makebox{\tt |\ Con}\ t_1\ \ldots \ t_n\ \makebox{\tt |}\ \ldots $}
\]
the corresponding definition of \mbox{\tt showsPrec} for \mbox{\tt Con} is shown in 
Figure~\ref{showsPrec-infix} for constructors declared in the infix
style and
Figure~\ref{showsPrec-fig} for all other constructors.  
See Appendix~\ref{stdprelude} for details of \mbox{\tt showParen}, \mbox{\tt showChar}, etc.

\begin{figure}
\outline{
\mbox{\tt \ \ \ \ showsPrec\ d\ (e1\ `Con`\ e2)\ =\ showParen\ (d\ >\ p)\ showStr}\\
\mbox{\tt \ \ \ \ \ \ where}\\
\mbox{\tt \ \ \ \ \ \ \ \ \ \ \ \ \ \ p\ =\ `the\ precedence\ of\ Con'}\\
\mbox{\tt \ \ \ \ \ \ \ \ \ \ \ \ \ \ lp\ =\ if\ `Con\ is\ left\ associative'\ then\ p\ else\ p+1}\\
\mbox{\tt \ \ \ \ \ \ \ \ \ \ \ \ \ \ rp\ =\ if\ `Con\ is\ right\ associative'\ then\ p\ else\ p+1}\\
\mbox{\tt \ \ \ \ \ \ \ \ \ \ \ \ \ \ cn\ =\ `the\ original\ name\ of\ Con'}\\
\mbox{\tt }\\[-8pt]
\mbox{\tt \ \ \ \ \ \ \ \ \ \ \ \ \ \ showStr\ =\ showsPrec\ lp\ e1\ .}\\
\mbox{\tt \ \ \ \ \ \ \ \ \ \ \ \ \ \ \ \ \ \ \ \ \ \ \ \ showChar\ '\ '\ .\ showString\ cn\ .\ showChar\ '\ '\ .}\\
\mbox{\tt \ \ \ \ \ \ \ \ \ \ \ \ \ \ \ \ \ \ \ \ \ \ \ \ showsPrec\ rp\ e2}
}
\caption{Specification of \mbox{\tt showsPrec} for Constructors Declared in the Infix Style}
\label{showsPrec-infix}
\end{figure}

\begin{figure}
\outline{
\mbox{\tt \ \ \ \ showsPrec\ d\ (Con\ e1\ ...\ en)\ =\ showParen\ (d\ >=\ 10)\ showStr}\\
\mbox{\tt \ \ \ \ \ \ where}\\
\mbox{\tt \ \ \ \ \ \ \ \ \ \ \ \ \ \ showStr\ =\ showString\ cn\ .\ showChar\ '\ '\ .}\\
\mbox{\tt \ \ \ \ \ \ \ \ \ \ \ \ \ \ \ \ \ \ \ \ \ \ \ \ showsPrec\ 10\ e1\ .\ showChar\ '\ '\ .}\\
\mbox{\tt \ \ \ \ \ \ \ \ \ \ \ \ \ \ \ \ \ \ \ \ \ \ \ \ ...}\\
\mbox{\tt \ \ \ \ \ \ \ \ \ \ \ \ \ \ \ \ \ \ \ \ \ \ \ \ showsPrec\ 10\ en}\\
\mbox{\tt \ \ \ \ \ \ \ \ \ \ \ \ \ \ cn\ =\ `the\ original\ name\ of\ Con'}
}
\caption{General Specification of \mbox{\tt showsPrec} for User-Defined Constructors}
\label{showsPrec-fig}
\end{figure}

\subsection{Specification of \mbox{\tt readsPrec}}
\label{readsPrec-spec}
\indextt{readsPrec}

A {\em lexeme} is exactly as in Section~\ref{lexical-structure}.
\mbox{\tt lex\ ::\ String\ ->\ [(String,String)]} reads the first lexeme from a
string.  If the string begins with a valid lexeme, the lexeme (with
leading whitespace removed) and the remainder of the string are
returned in a singleton list.  If no lexeme is present or the lexeme
is not syntacticly correct, \mbox{\tt []} is returned.  A full definition is
provided in Appendix~\ref{preludetext}.

% \mbox{\tt lex\ ::\ String\ ->\ [(String,\ String)]} parses a string into two
% parts: (1)~a string representing the first lexeme or \mbox{\tt ""} if the input
% is \mbox{\tt ""} or consists entirely of white space, and (2)~the remainder of
% the input after the first lexeme is extracted.  Whitespace (again
% refer to Section~\ref{lexical-structure}) is ignored except within
% strings.  A successful parse results in a singleton list containing
% a pair of strings; if no proper lexeme can be parsed (such as in the case
% of an unrecognised control character), an empty list is returned.
% A full definition is provided in Appendix~\ref{preludetext}.

As described in Section~\ref{derived-decls}, \mbox{\tt readsPrec} has the type
\[
\mbox{$\it \makebox{\tt Text\ a\ =>\ Int\ ->\ String\ ->\ [(a,String)]}$}
\]
Its first parameter is a
precedence in the range 0 to 10, its second is the string to be
parsed.  
Figure~\ref{readsPrec-fig} shows the specification of \mbox{\tt readsPrec} for user-defined 
datatypes of the form:
\[
\mbox{$\it \makebox{\tt data}\ c\ \makebox{\tt =>}\ T\ u_1\ \ldots \ u_k\ \makebox{\tt =}\ K_1\ t_{11}\ \ldots \ t_{1k_1}\ \makebox{\tt |}\ \ldots \ \makebox{\tt |}\ K_n\ t_{n1}\ \ldots \ t_{nk_n}$}
\]

\begin{figure}
\outline{
\mbox{\tt readsPrec\ d\ r\ =\ readCon\ K1\ k1\ `the\ original\ name\ of\ K1'\ r\ ++}\\
\mbox{\tt \ \ \ \ \ \ \ \ \ \ \ ...\ }\\
\mbox{\tt \ \ \ \ \ \ \ \ \ \ \ readCon\ Kn\ kn\ `the\ original\ name\ of\ Kn'\ r}\\
\mbox{\tt \ \ where}\\
\mbox{\tt \ \ \ \ readCon\ con\ n\ cn\ =\ \ \ \ \ \ \ \ \ \ \ \ \ \ \ \ \ \ --\ if\ con\ is\ infix}\\
\mbox{\tt \ \ \ \ \ \ \ \ readParen\ (d\ >\ p)\ readVal\ }\\
\mbox{\tt \ \ \ \ \ \ where}\\
\mbox{\tt \ \ \ \ \ \ \ \ \ \ \ \ \ \ \ \ readVal\ r\ =\ \ [(u\ `con`\ v,\ s2)\ |}\\
\mbox{\tt \ \ \ \ \ \ \ \ \ \ \ \ \ \ \ \ \ \ \ \ \ \ \ \ \ \ \ \ \ (u,s0)\ \ \ <-\ readsPrec\ lp\ r,}\\
\mbox{\tt \ \ \ \ \ \ \ \ \ \ \ \ \ \ \ \ \ \ \ \ \ \ \ \ \ \ \ \ \ (tok,s1)\ <-\ lex\ s0,\ tok\ ==\ cn,}\\
\mbox{\tt \ \ \ \ \ \ \ \ \ \ \ \ \ \ \ \ \ \ \ \ \ \ \ \ \ \ \ \ \ (v,s2)\ \ \ <-\ readsPrec\ rp\ s1]}\\
\mbox{\tt \ \ \ \ \ \ \ \ \ \ \ \ \ \ \ \ p\ =\ `the\ precedence\ of\ con'}\\
\mbox{\tt \ \ \ \ \ \ \ \ \ \ \ \ \ \ \ \ lp\ =\ if\ `con\ is\ left\ associative'\ then\ p\ else\ p+1}\\
\mbox{\tt \ \ \ \ \ \ \ \ \ \ \ \ \ \ \ \ rp\ =\ if\ `con\ is\ right\ associative'\ then\ p\ else\ p+1}\\
\mbox{\tt }\\[-8pt]
\mbox{\tt \ \ \ \ readCon\ con\ n\ cn\ =\ \ \ \ \ \ \ \ \ \ \ \ \ \ \ \ \ \ --\ if\ con\ is\ not\ infix}\\
\mbox{\tt \ \ \ \ \ \ \ \ readParen\ (d\ >\ 9)\ readVal}\\
\mbox{\tt \ \ \ \ \ \ where}\\
\mbox{\tt \ \ \ \ \ \ \ \ \ \ \ \ \ \ \ \ \ readVal\ r\ =\ [(con\ t1\ ...\ tn,\ sn)\ |}\\
\mbox{\tt \ \ \ \ \ \ \ \ \ \ \ \ \ \ \ \ \ \ \ \ \ \ \ \ \ \ \ \ \ (t0,s0)\ <-\ lex\ r,\ t0\ ==\ cn,}\\
\mbox{\tt \ \ \ \ \ \ \ \ \ \ \ \ \ \ \ \ \ \ \ \ \ \ \ \ \ \ \ \ \ (t1,s1)\ <-\ readsPrec\ 10\ s0,}\\
\mbox{\tt \ \ \ \ \ \ \ \ \ \ \ \ \ \ \ \ \ \ \ \ \ \ \ \ \ \ \ \ \ ...}\\
\mbox{\tt \ \ \ \ \ \ \ \ \ \ \ \ \ \ \ \ \ \ \ \ \ \ \ \ \ \ \ \ \ (tn,sn)\ <-\ readsPrec\ 10\ s(n-1)]}
}
\caption{Definition of \mbox{\tt readsPrec} for User-Defined Types}
\label{readsPrec-fig}
\end{figure}

\subsection{An example}

As a complete example, consider a tree datatype:\nopagebreak
%\bprog
%@
%data Tree a = Leaf a | Tree a :^: Tree a
%@
%\eprog\nopagebreak
%Since there is no \mbox{\tt deriving} clause, this is shorthand for:\nopagebreak
\bprog
\mbox{\tt data\ Tree\ a\ =\ Leaf\ a\ |\ Tree\ a\ :{\char'136}:\ Tree\ a}\\
\mbox{\tt \ \ \ \ \ deriving\ (Eq,\ Ord,\ Text,\ Binary)}\\
\mbox{\tt instance\ (Eq\ a)\ =>\ Eq\ (Tree\ a)}\\
\mbox{\tt \ \ where\ ...}\\
\mbox{\tt instance\ (Ord\ a)\ =>\ Ord\ (Tree\ a)}\\
\mbox{\tt \ \ where\ ...}\\
\mbox{\tt instance\ (Text\ a)\ =>\ Text\ (Tree\ a)}\\
\mbox{\tt \ \ where\ ...}\\
\mbox{\tt instance\ (Binary\ a)\ =>\ Binary\ (Tree\ a)}\\
\mbox{\tt \ \ where\ ...}
\eprog
Note the recursive context; the components of the datatype must
themselves be instances of the class.  Automatic derivation of
instance
declarations for \mbox{\tt Ix} and \mbox{\tt Enum} are not possible, as \mbox{\tt Tree} is not
an enumeration or single-constructor datatype.  Except for \mbox{\tt Binary}, the complete
instance declarations for \mbox{\tt Tree} are shown in Figure~\ref{tree-inst},
Note the implicit use of default-method
\index{default method}
definitions---for
example, only \mbox{\tt <=} is defined for \mbox{\tt Ord}, with the other
operations (\mbox{\tt <}, \mbox{\tt >}, \mbox{\tt >=}, \mbox{\tt max}, and \mbox{\tt min}) being defined by the defaults given in
the class declaration shown in Figure~\ref{standard-classes}
(page~\pageref{standard-classes}).

\begin{figure}
\outline{
\mbox{\tt infix\ 4\ :{\char'136}:}\\
\mbox{\tt data\ Tree\ a\ =\ \ Leaf\ a\ \ |\ \ Tree\ a\ :{\char'136}:\ Tree\ a}\\
\mbox{\tt }\\[-8pt]
\mbox{\tt instance\ (Eq\ a)\ =>\ Eq\ (Tree\ a)\ where}\\
\mbox{\tt \ \ \ \ \ \ \ \ Leaf\ m\ ==\ Leaf\ n\ \ =\ \ m==n}\\
\mbox{\tt \ \ \ \ \ \ \ \ u:{\char'136}:v\ \ ==\ x:{\char'136}:y\ \ \ =\ \ u==x\ {\char'46}{\char'46}\ v==y}\\
\mbox{\tt \ \ \ \ \ \ \ \ \ \ \ \ \ {\char'137}\ ==\ {\char'137}\ \ \ \ \ \ \ =\ \ False}\\
\mbox{\tt }\\[-8pt]
\mbox{\tt instance\ (Ord\ a)\ =>\ Ord\ (Tree\ a)\ where}\\
\mbox{\tt \ \ \ \ \ \ \ \ Leaf\ m\ <=\ Leaf\ n\ \ =\ \ m<=n}\\
\mbox{\tt \ \ \ \ \ \ \ \ Leaf\ m\ <=\ x:{\char'136}:y\ \ \ =\ \ True}\\
\mbox{\tt \ \ \ \ \ \ \ \ u:{\char'136}:v\ \ <=\ Leaf\ n\ \ =\ \ False}\\
\mbox{\tt \ \ \ \ \ \ \ \ u:{\char'136}:v\ \ <=\ x:{\char'136}:y\ \ \ =\ \ u<x\ ||\ u==x\ {\char'46}{\char'46}\ v<=y}\\
\mbox{\tt }\\[-8pt]
\mbox{\tt instance\ (Text\ a)\ =>\ Text\ (Tree\ a)\ where}\\
\mbox{\tt }\\[-8pt]
\mbox{\tt \ \ \ \ \ \ \ \ showsPrec\ d\ (Leaf\ m)\ =\ showParen\ (d\ >=\ 10)\ showStr}\\
\mbox{\tt \ \ \ \ \ \ \ \ \ \ where}\\
\mbox{\tt \ \ \ \ \ \ \ \ \ \ \ \ \ showStr\ =\ showString\ "Leaf"\ .\ showChar\ '\ '\ .\ showsPrec\ 10\ m}\\
\mbox{\tt }\\[-8pt]
\mbox{\tt \ \ \ \ \ \ \ \ showsPrec\ d\ (u\ :{\char'136}:\ v)\ =\ showParen\ (d\ >\ 4)\ showStr}\\
\mbox{\tt \ \ \ \ \ \ \ \ \ \ where}\\
\mbox{\tt \ \ \ \ \ \ \ \ \ \ \ \ \ showStr\ =\ showsPrec\ 5\ u\ .\ }\\
\mbox{\tt \ \ \ \ \ \ \ \ \ \ \ \ \ \ \ \ \ \ \ \ \ \ \ showChar\ '\ '\ .\ showString\ ":{\char'136}:"\ .\ showChar\ '\ '\ .}\\
\mbox{\tt \ \ \ \ \ \ \ \ \ \ \ \ \ \ \ \ \ \ \ \ \ \ \ showsPrec\ 5\ v}\\
\mbox{\tt }\\[-8pt]
\mbox{\tt \ \ \ \ \ \ \ \ readsPrec\ d\ r\ =\ \ readParen\ (d\ >\ 4)}\\
\mbox{\tt \ \ \ \ \ \ \ \ \ \ \ \ \ \ \ \ \ \ \ \ \ \ \ \ \ ({\char'134}r\ ->\ [(u:{\char'136}:v,w)\ |}\\
\mbox{\tt \ \ \ \ \ \ \ \ \ \ \ \ \ \ \ \ \ \ \ \ \ \ \ \ \ \ \ \ \ \ \ \ \ (u,s)\ <-\ readsPrec\ 5\ r,}\\
\mbox{\tt \ \ \ \ \ \ \ \ \ \ \ \ \ \ \ \ \ \ \ \ \ \ \ \ \ \ \ \ \ \ \ \ \ (":{\char'136}:",t)\ <-\ lex\ s,}\\
\mbox{\tt \ \ \ \ \ \ \ \ \ \ \ \ \ \ \ \ \ \ \ \ \ \ \ \ \ \ \ \ \ \ \ \ \ (v,w)\ <-\ readsPrec\ 5\ t])\ r}\\
\mbox{\tt }\\[-8pt]
\mbox{\tt \ \ \ \ \ \ \ \ \ \ \ \ \ \ \ \ \ \ \ \ \ \ ++\ readParen\ (d\ >\ 9)}\\
\mbox{\tt \ \ \ \ \ \ \ \ \ \ \ \ \ \ \ \ \ \ \ \ \ \ \ \ \ ({\char'134}r\ ->\ [(Leaf\ m,t)\ |}\\
\mbox{\tt \ \ \ \ \ \ \ \ \ \ \ \ \ \ \ \ \ \ \ \ \ \ \ \ \ \ \ \ \ \ \ \ \ ("Leaf",t)\ <-\ lex\ r,}\\
\mbox{\tt \ \ \ \ \ \ \ \ \ \ \ \ \ \ \ \ \ \ \ \ \ \ \ \ \ \ \ \ \ \ \ \ \ (m,t)\ <-\ readsPrec\ 10\ t])\ r}
}
\ecaption{Example of Derived Instances}
\label{tree-inst}
\end{figure}

% Local Variables: 
% mode: latex
% End:
