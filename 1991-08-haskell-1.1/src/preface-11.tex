%
% $Header$
%
\begin{center}
\Large\bf Preface to Version~1.1
\end{center}

\vspace{.2in}

\noindent
Following the Version~1.0 \Haskell{}~report, several sites have
implemented \Haskell{} (or a subset thereof) and people have started
to use these implementations.  Based on this experience of
implementation and use, it became apparent that a modest revision of
the language would be desirable, in which some improvements in syntax
could be made and certain features generalised.  This Version~1.1
report is the result.

This revision was specifically {\em not} intended to add any
substantial new features to the language, but rather to ``tidy up''
the existing language.  Despite this narrow focus, a wide
debate ensued, conducted on the \Haskell{} mailing list
\index{Haskell mailing list@\Haskell{} mailing list}
(see page~\pageref{haskell-mailing-list}) rather than just among members of
the original committee.

In this minor revision, the tricky issues identified in the preface to
Version~1.0 remain, so that preface should be read in conjunction with
this one.

\subsection*{Summary of changes}
\label{preface-changes-11}

The main changes (other than concrete syntax) are as follows.

\begin{itemize}
\item
Class methods
\index{class method}
may be polymorphic and overloaded in type variables
other than the class variable (Section~\ref{class-decls}).

\item
The ``monomorphism restriction''
\index{monomorphism restriction}
has been made more precise, and
relaxed in the case where the programmer supplies a type signature
(Section~\ref{function-bindings}).

\item
The meaning of contexts in \mbox{\tt data} declarations has been clarified
\index{context!in data declaration@in {\ptt data} declaration}
(Section~\ref{datatype-decls}), and \mbox{\tt type} synonym declarations are no
longer permitted to have contexts (Section~\ref{type-synonym-decls}).
\index{type synonym}

\item
If the \mbox{\tt deriving} clause on a \mbox{\tt data} declaration is omitted, no
instances are automatically derived (Section~\ref{derived-decls}).
\index{derived instance}

\item
A module \mbox{$\it m$} may refer to all of its own local definitions in an export
list using \mbox{$\it m\makebox{\tt ..}$} (Section~\ref{export}).
\index{export list}
\end{itemize}

\noindent
The main syntactic changes are as follows:
\begin{itemize}
\item
A new form of expression, a \mbox{\tt let}-expression,
\index{let expression}
has been added, which replaces and has the same semantics as a \mbox{\tt where}
expression.  (In particular, the bindings it introduces are mutually
recursive; \Haskell{} has no non-recursive \mbox{\tt let} construct.)
Bindings may also be introduced by a \mbox{\tt where} clause, but such
\mbox{\tt where} clauses are now attached to a group of guarded right-hand
sides, and scope over the guards.  The previous inability to scope
definitions over guards was a significant shortcoming of the language.

\item
Sections
\index{section}
have been introduced for binary operators.  For example, the
expression \mbox{\tt (/\ 2)} is the function which divides its argument by 2,
and \mbox{\tt (2\ /)} is the function which divides 2 by its argument.

\item
The standard prelude has been debugged and revised.

%\item
%The syntax of interfaces has been changed, to make it clearer
%what is being {\em exported} and what is merely being {\em named}.

\end{itemize}

\noindent
A few other nontrivial changes to the syntax are listed in
Appendix~\ref{syntax-changes}.

\subsection*{Implementations}\label{implementors}

Several groups are working on implementations of \Haskell{}, including
those at Chalmers (contact: \mbox{\tt hbc@cs.chalmers.se}), Glasgow
(\mbox{\tt haskell-request@dcs.glasgow.ac.uk}), Syracuse (\mbox{\tt polar@top.cis.syr.edu}), and Yale
(\mbox{\tt haskell-request@cs.yale.edu}).  Official announcements about these
implementations will appear on the \Haskell{} technical mailing list
\index{Haskell mailing list@\Haskell{} mailing list}
(see page~\pageref{haskell-mailing-list}).

\subsection*{Formal Semantics}
\index{formal semantics}

Work has also been undertaken at Glasgow on a formal static and
dynamic semantics for \Haskell{}
\cite{dynamic-semantics,static-semantics}.  These efforts are well
advanced but as yet incomplete.

\subsection*{Acknowledgements}

Language design is an evolutionary process, and the group of people
involved undergoes evolution as well.  We wish to thank past members
of the \Haskell{} Committee---Arvind, Mike Reeve, David Wise, and
Jonathan Young---for their previous contributions and continued
support.  We also thank those who braved the storm of electronic mail
on the \Haskell{} mailing list, and responded with constructive
suggestions for the revised language.  The following were especially
helpful and active:
Lennart Augustsson,
Cordelia Hall,
Kent Karlsson,
Mark Jones,
Mark Lillibridge,
and Satish Thatte.  

%\ToDo{past authors sentence (Simon)}

Numerous others contributed to the debate, and we thank them also.

\begin{flushright}
The \Haskell{} Committee\\
19 August 1991
\end{flushright}

% Local Variables: 
% mode: latex
% End:

