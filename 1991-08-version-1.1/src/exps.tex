%
% $Header$
%
\section{Expressions}\index{expression}
\label{expressions}

In this section, we describe the syntax and informal semantics of \Haskell{}
{\em expressions}, including their translations into the
\Haskell{} kernel, where appropriate.
% (see Appendix~\ref{formal-semantics} for a formal syntax and
% semantics of the kernel).

\begin{flushleft}\it\begin{tabbing}
\hspace{0.5in}\=\hspace{3.0in}\=\kill
$\it exp$\>\makebox[3.5em]{$\rightarrow$}$\it \makebox{\tt {\char'134}}\ apat_1\ \ldots \ apat_n\ \makebox{\tt ->}\ exp$\>\makebox[3em]{}$\it (\tr{lambda\ abstraction},\ n\geq 1)$\\ 
$\it $\>\makebox[3.5em]{$|$}$\it \makebox{\tt let}\ \makebox{\tt {\char'173}}\ decls\ [\makebox{\tt ;}]\ \makebox{\tt {\char'175}}\ \makebox{\tt in}\ exp$\>\makebox[3em]{}$\it ({\tr{let\ expression}})$\\ 
$\it $\>\makebox[3.5em]{$|$}$\it \makebox{\tt if}\ exp\ \makebox{\tt then}\ exp\ \makebox{\tt else}\ exp$\>\makebox[3em]{}$\it (\tr{conditional})$\\ 
$\it $\>\makebox[3.5em]{$|$}$\it \makebox{\tt case}\ exp\ \makebox{\tt of}\ \makebox{\tt {\char'173}}\ alts\ [\makebox{\tt ;}]\ \makebox{\tt {\char'175}}$\>\makebox[3em]{}$\it (\tr{case\ expression})$\\ 
$\it $\>\makebox[3.5em]{$|$}$\it exp^0\ \makebox{\tt ::}\ [context\ \makebox{\tt =>}]\ atype$\>\makebox[3em]{}$\it (\tr{expression\ type\ signature})$\\ 
$\it $\>\makebox[3.5em]{$|$}$\it exp^0$\\ 
$\it exp^i$\>\makebox[3.5em]{$\rightarrow$}$\it exp^{i+1}\ [op^{({\rm\ n},i)}\ exp^{i+1}]$\>\makebox[3em]{}$\it (0\leq i\leq 9)$\\ 
$\it $\>\makebox[3.5em]{$|$}$\it lexp^i\ op^{({\rm\ l},i)}\ exp^{i+1}$\\ 
$\it $\>\makebox[3.5em]{$|$}$\it exp^{i+1}\ op^{({\rm\ r},i)}\ rexp^i$\\ 
$\it lexp^i$\>\makebox[3.5em]{$\rightarrow$}$\it [lexp^i\ op^{({\rm\ l},i)}]\ exp^{i+1}$\>\makebox[3em]{}$\it (0\leq i\leq 9)$\\ 
$\it lexp^6$\>\makebox[3.5em]{$\rightarrow$}$\it \makebox{\tt -}\ exp^7$\\ 
$\it rexp^i$\>\makebox[3.5em]{$\rightarrow$}$\it exp^{i+1}\ [op^{({\rm\ r},i)}\ rexp^i]$\>\makebox[3em]{}$\it (0\leq i\leq 9)$\\ 
$\it exp^{10}$\>\makebox[3.5em]{$\rightarrow$}$\it exp^{10}\ aexp$\>\makebox[3em]{}$\it (\tr{function\ application})$\\ 
$\it $\>\makebox[3.5em]{$|$}$\it aexp$\\ 
$\it $\\ 
$\it aexp$\>\makebox[3.5em]{$\rightarrow$}$\it var$\>\makebox[3em]{}$\it (\tr{variable})$\\ 
$\it $\>\makebox[3.5em]{$|$}$\it con$\>\makebox[3em]{}$\it (\tr{constructor})$\\ 
$\it $\>\makebox[3.5em]{$|$}$\it literal$\\ 
$\it $\>\makebox[3.5em]{$|$}$\it \makebox{\tt ()}$\>\makebox[3em]{}$\it (\tr{unit})$\\ 
$\it $\>\makebox[3.5em]{$|$}$\it \makebox{\tt (}\ exp\ \makebox{\tt )}$\>\makebox[3em]{}$\it (\tr{parenthesised\ expression})$\\ 
$\it $\>\makebox[3.5em]{$|$}$\it \makebox{\tt (}\ exp_1\ \makebox{\tt ,}\ \ldots \ \makebox{\tt ,}\ exp_k\ \makebox{\tt )}$\>\makebox[3em]{}$\it (\tr{tuple},\ k\geq 2)$\\ 
$\it $\>\makebox[3.5em]{$|$}$\it \makebox{\tt [}\ exp_1\ \makebox{\tt ,}\ \ldots \ \makebox{\tt ,}\ exp_k\ \makebox{\tt ]}$\>\makebox[3em]{}$\it (\tr{list},\ k\geq 0)$\\ 
$\it $\>\makebox[3.5em]{$|$}$\it \makebox{\tt [}\ exp_1\ [\makebox{\tt ,}\ exp_2]\ \makebox{\tt ..}\ [exp_3]\ \makebox{\tt ]}$\>\makebox[3em]{}$\it (\tr{arithmetic\ sequence})$\\ 
$\it $\>\makebox[3.5em]{$|$}$\it \makebox{\tt [}\ exp\ \makebox{\tt |}\ qual_1\ \makebox{\tt ,}\ \ldots \ \makebox{\tt ,}\ qual_n\ \makebox{\tt ]}$\>\makebox[3em]{}$\it (\tr{list\ comprehension},\ n\geq 1)$\\ 
$\it $\>\makebox[3.5em]{$|$}$\it \makebox{\tt (}\ exp^{i+1}\ op^{(a,i)}\ \makebox{\tt )}$\>\makebox[3em]{}$\it (0\ \leq i\leq \ 9,\ a\in\{l,r,n\})$\\ 
$\it $\>\makebox[3.5em]{$|$}$\it \makebox{\tt (}\ op^{(a,i)}\ exp^{i+1}\ \makebox{\tt )}$\>\makebox[3em]{}$\it (0\ \leq i\leq \ 9,\ a\in\{l,r,n\})$\\ 
$\it %\ |\ \makebox{\tt (}\ lexp^i\ op^{({\rm\ l},i)}\ \makebox{\tt )}$\>\makebox[3em]{}$\it (0\ \leq \ i\ \leq \ 9)$\\ 
$\it %\ |\ \makebox{\tt (}\ op^{({\rm\ r},i)}\ rexp^i\ \makebox{\tt )}$\>\makebox[3em]{}$\it (0\ \leq \ i\ \leq \ 9)$\\ 
$\it $\\ 
$\it op$\>\makebox[3.5em]{$\rightarrow$}$\it varop\ |\ conop$
\end{tabbing}\end{flushleft}
\indexsyn{exp}%
\index{exp@\mbox{$\it exp^i$}}%
\index{lexp@\mbox{$\it lexp^i$}}%
\index{rexp@\mbox{$\it rexp^i$}}%
\indexsyn{aexp}%
\indexsyn{op}%

The grammar above embodies the following precedence hierarchy for
expressions, using productions with superscripts as described in
Section~\ref{notational-conventions}:
\begin{itemize}
\item
Function application binds most tightly of all.
\item
Expressions involving infix operators are disambiguated by the
operator's fixity (see Section~\ref{fixity}).
Consecutive unparenthesised operators with the same
precedence must both be either left or right associative
to avoid a syntax error.
Given an unparenthesised expression ``\mbox{$\it x\ op^{(a,i)}\ y\ op^{(b,j)}\ z$}'', parentheses
must be added around either ``\mbox{$\it x\ op^{(a,i)}\ y$}'' or ``\mbox{$\it y\ op^{(b,j)}\\
\it z$}'' when \mbox{$\it i=j$} unless \mbox{$\it a=b={\rm\ l}$} or \mbox{$\it a=b={\rm\ r}$}.
\item
Negation\index{negation} is the only prefix operator in
\Haskell{}; it has the same precedence as the infix \mbox{\tt -} operator
defined in the standard prelude (see Figure~\ref{prelude-fixities}).
\item
Expression type signatures are parsed as if \mbox{\tt ::} were a
left-associative infix operator with precedence lower than any other
operator.
\end{itemize}
Sample parses using this grammar are shown below.
\[\bt{|l|l|}\hline
This                                & Parses as                             \\
\hline
\mbox{\tt f\ x\ +\ g\ y}                         & \mbox{\tt (f\ x)\ +\ (g\ y)}                       \\
\mbox{\tt -\ f\ x\ +\ y}                         & \mbox{\tt (-\ (f\ x))\ +\ y}                       \\
\mbox{\tt let\ {\char'173}\ ...\ {\char'175}\ in\ x\ +\ y}              & \mbox{\tt let\ {\char'173}\ ...\ {\char'175}\ in\ (x\ +\ y)}              \\
\mbox{\tt f\ x\ y\ ::\ Int}                      & \mbox{\tt (f\ x\ y)\ ::\ Int}                      \\
\mbox{\tt {\char'134}\ x\ ->\ a+b\ ::\ Int}                 & \mbox{\tt {\char'134}\ x\ ->\ ((a+b)\ ::\ Int})               \\
\hline\et\]

For the sake of clarity, the rest of this section shows the syntax of
expressions without their precedences.

\subsection{Curried Applications and Lambda Abstractions}
\label{applications}
\label{lambda-abstractions}
\index{lambda abstraction}
\index{\ pats -> expr@\mbox{$\it \makebox{\tt {\char'134}}\ pats\ \makebox{\tt ->}\ expr$}}
\index{application}
%\index{function application|see{application}}
%
\begin{flushleft}\it\begin{tabbing}
\hspace{0.5in}\=\hspace{3.0in}\=\kill
$\it exp$\>\makebox[3.5em]{$\rightarrow$}$\it exp\ aexp$\\ 
$\it exp$\>\makebox[3.5em]{$\rightarrow$}$\it \makebox{\tt {\char'134}}\ apat_1\ \ldots \ apat_n\ \makebox{\tt ->}\ exp$
\end{tabbing}\end{flushleft}
\indexsyn{exp}%
%
\noindent
{\em Function application}\index{application} is written 
\mbox{$\it e_1\ e_2$}.  Application associates to the left, so the
parentheses may be omitted in \mbox{\tt (f\ x)\ y}, for example.  Because \mbox{$\it e_1$} could
be a constructor, partial applications of constructors are allowed.

{\em Lambda abstractions} are written 
\mbox{$\it \makebox{\tt {\char'134}}\ p_1\ \ldots \ p_n\ \makebox{\tt ->}\ e$}, where the \mbox{$\it p_i$} are {\em patterns}.
An expression such as \mbox{\tt {\char'134}x:xs->x} is syntactically incorrect,
and must be rewritten as \mbox{\tt {\char'134}(x:xs)->x}.

The set of patterns must be {\em linear}\index{linearity}
\index{linear pattern}---no variable may appear more than once in the set.

\outline{
\paragraph*{Translation:}
The lambda abstraction \mbox{$\it \makebox{\tt {\char'134}}\ p_1\ \ldots \ p_n\ \makebox{\tt ->}\ e$} is equivalent to
\[
\mbox{$\it \makebox{\tt {\char'134}}\ x_1\ \ldots \ x_n\ \makebox{\tt ->\ case\ (}x_1\makebox{\tt ,}\ \ldots \makebox{\tt ,}\ x_n\makebox{\tt )\ of\ (}p_1\makebox{\tt ,}\ \ldots \makebox{\tt ,}\ p_n\makebox{\tt )\ ->}\ e$}
\]
where the \mbox{$\it x_i$} are new identifiers.
Given this translation combined with the semantics of case
expressions and pattern-matching described in
Section~\ref{case-semantics}, if the
pattern fails to match, then the result is \mbox{$\it \bot$}.
}
             
The type of a variable bound by a lambda abstraction is monomorphic,
\index{monomorphic type variable}
as is always the case in the Hindley-Milner type system.

\subsection{Operator Applications}
\index{operator application}
%\index{operator application|seealso{application}}
\label{operators}
%
\begin{flushleft}\it\begin{tabbing}
\hspace{0.5in}\=\hspace{3.0in}\=\kill
$\it exp$\>\makebox[3.5em]{$\rightarrow$}$\it exp_1\ op\ exp_2$\\ 
$\it $\>\makebox[3.5em]{$|$}$\it \makebox{\tt -}\ exp$\>\makebox[3em]{}$\it (prefix\ negation)$
\end{tabbing}\end{flushleft}
\indexsyn{exp}%
\noindent
The form \mbox{$\it e_1\ op\ e_2$} is the infix application of binary
operator\index{operator} \mbox{$\it op$} to expressions \mbox{$\it e_1$} and \mbox{$\it e_2$}.  

The special
form \mbox{$\it \makebox{\tt -}e$} denotes prefix negation\index{negation}, the one and only
prefix operator in \Haskell{}, and is simply
syntax for \mbox{$\it \makebox{\tt negate\ }(e)$},\indextt{negate} where \mbox{\tt negate} is as
defined in the standard prelude (see
Figure~\ref{basic-numeric-1},
page~\pageref{basic-numeric-1}).
Because \mbox{\tt e1-e2} parses as an
infix application of the binary operator \mbox{\tt -}, one must write \mbox{\tt e1(-e2)} for
the alternative parsing.  Similarly, \mbox{\tt (-)} is syntax for 
\mbox{\tt ({\char'134}\ x\ y\ ->\ x-y)}, as with any infix operator, and does not denote 
\mbox{\tt ({\char'134}\ x\ ->\ -x)}---one must use \mbox{\tt negate} for that.

\outline{
\paragraph*{Translation:}
\mbox{$\it e_1\ op\ e_2$} is equivalent to \mbox{$\it \makebox{\tt (}op\makebox{\tt )}\ e_1\ e_2$}.  \mbox{$\it \makebox{\tt -}e$} is
equivalent to \mbox{$\it \makebox{\tt \ negate}\ (e)$} where \mbox{\tt negate}, an operator in the class
\mbox{\tt Num}, is as defined in the standard prelude.
}

\subsection{Sections}
\index{section}
%\index{section|seealso{operator application}}
\label{sections}
%
\begin{flushleft}\it\begin{tabbing}
\hspace{0.5in}\=\hspace{3.0in}\=\kill
$\it aexp$\>\makebox[3.5em]{$\rightarrow$}$\it \makebox{\tt (}\ exp\ op\ \makebox{\tt )}$\\ 
$\it $\>\makebox[3.5em]{$|$}$\it \makebox{\tt (}\ op\ exp\ \makebox{\tt )}$
\end{tabbing}\end{flushleft}
\indexsyn{aexp}%
\noindent
{\em Sections} are written as \mbox{$\it \makebox{\tt (}\ op\ e\ \makebox{\tt )}$} or \mbox{$\it \makebox{\tt (}\ e\ op\ \makebox{\tt )}$}, where
\mbox{$\it op$} is a binary operator and \mbox{$\it e$} is an expression.  Sections are a
convenient syntax for partial application of binary operators.

The normal rules of syntactic precedence apply to sections; for
example, \mbox{\tt (*a+b)} is syntactically invalid, but \mbox{\tt (+a*b)} and
\mbox{\tt (*(a+b))} are valid.  Syntactic associativity, however, is not
taken into account in sections; thus, \mbox{\tt (a+b+)} must be written
\mbox{\tt ((a+b)+)}.

Because \mbox{\tt -} is treated specially in the grammar,
\mbox{$\it \makebox{\tt (-}\ exp\makebox{\tt )}$} is not a section, but an application of prefix
negation,\index{negation} as
described in the preceding section.  However, there is a \mbox{\tt subtract}
function defined in the standard prelude such that
\mbox{$\it \makebox{\tt (subtract}\ exp\makebox{\tt )}$} is equivalent to the disallowed section.
The expression \mbox{$\it \makebox{\tt (+\ (-}\ exp\makebox{\tt ))}$} can serve the same purpose.

\outline{
\paragraph*{Translation:}
For binary operator \mbox{$\it op$} and expression \mbox{$\it e$}, if \mbox{$\it x$} is a variable
that does not occur free in \mbox{$\it e$}, the section
\mbox{$\it \makebox{\tt (}op\ e\makebox{\tt )}$} is equivalent to \mbox{$\it \makebox{\tt {\char'134}}\ x\ \makebox{\tt ->}\ x\ op\ e$}, and the section 
\mbox{$\it \makebox{\tt (}e\ op\makebox{\tt )}$} is equivalent to \mbox{$\it \makebox{\tt {\char'134}}\ x\ \makebox{\tt ->}\ e\ op\ x$}.
}

\subsection{Conditionals}
\label{conditionals}\index{conditional expression}
%
\begin{flushleft}\it\begin{tabbing}
\hspace{0.5in}\=\hspace{3.0in}\=\kill
$\it exp$\>\makebox[3.5em]{$\rightarrow$}$\it \makebox{\tt if}\ exp_1\ \makebox{\tt then}\ exp_2\ \makebox{\tt else}\ exp_3$
\end{tabbing}\end{flushleft}
\indexsyn{exp}%
%\indextt{if ... then ... else ...}
A {\em conditional expression}
\index{conditional expression}
has the form 
\mbox{$\it \makebox{\tt if}\ e_1\ \makebox{\tt then}\ e_2\ \makebox{\tt else}\ e_3$} and returns the value of \mbox{$\it e_2$} if the
value of \mbox{$\it e_1$} is \mbox{\tt True}, \mbox{$\it e_3$} if \mbox{$\it e_1$} is \mbox{\tt False}, and \mbox{$\it \bot$}
otherwise.

\outline{
\paragraph*{Translation:}
\mbox{$\it \makebox{\tt if}\ e_1\ \makebox{\tt then}\ e_2\ \makebox{\tt else}\ e_3$} is equivalent to:
\[
\mbox{$\it \makebox{\tt case}\ e_1\ \makebox{\tt of\ {\char'173}\ True\ ->}\ e_2\ \makebox{\tt ;\ False\ ->}\ e_3\ \makebox{\tt {\char'175}}$}
\]
where \mbox{\tt True} and \mbox{\tt False} are the two nullary constructors from the
type \mbox{\tt Bool}, as defined in the standard prelude.
}

\subsection{Lists}
\label{lists}
%
\begin{flushleft}\it\begin{tabbing}
\hspace{0.5in}\=\hspace{3.0in}\=\kill
$\it aexp$\>\makebox[3.5em]{$\rightarrow$}$\it \makebox{\tt [}\ exp_1\ \makebox{\tt ,}\ \ldots \ \makebox{\tt ,}\ exp_k\ \makebox{\tt ]}$\>\makebox[3em]{}$\it (k\geq 0)$
\end{tabbing}\end{flushleft}
\indexsyn{aexp}%
\index{[e1,...en]@\mbox{$\it \makebox{\tt [}e_1,\ldots ,e_n\makebox{\tt ]}$} (list)}
%
{\em Lists}\index{list} are written \mbox{$\it \makebox{\tt [}e_1\makebox{\tt ,}\ \ldots \makebox{\tt ,}\ e_k\makebox{\tt ]}$}, where
\mbox{$\it k\geq 0$}; the empty list is written \mbox{\tt []}.  Standard operations on
lists are given in the standard prelude (see
Appendix~\ref{stdprelude}, notably Section~\ref{preludelist}).

\outline{
\paragraph*{Translation:}  
\mbox{$\it \makebox{\tt [}e_1\makebox{\tt ,}\ \ldots \makebox{\tt ,}\ e_k\makebox{\tt ]}$} is equivalent to
\[
\mbox{$\it e_1\ \makebox{\tt :\ (}e_2\ \makebox{\tt :\ (}\ \ldots \ \makebox{\tt (}e_k\ \makebox{\tt :\ [])))}$}
\]
where \mbox{\tt :} and \mbox{\tt []} are constructors for lists, as defined in
the standard prelude (see Section~\ref{basic-lists}).  The types
of \mbox{$\it e_1$} through \mbox{$\it e_k$} must all be the same (call it \mbox{$\it t\/$}), and the
type of the overall expression is \mbox{\tt [}\mbox{$\it t$}\mbox{\tt ]} (see Section~\ref{type-syntax}).
}

\subsection{Tuples}
\label{tuples}
%
\begin{flushleft}\it\begin{tabbing}
\hspace{0.5in}\=\hspace{3.0in}\=\kill
$\it aexp$\>\makebox[3.5em]{$\rightarrow$}$\it \makebox{\tt (}\ exp_1\ \makebox{\tt ,}\ \ldots \ \makebox{\tt ,}\ exp_k\ \makebox{\tt )}$\>\makebox[3em]{}$\it (k\geq 2)$
\end{tabbing}\end{flushleft}
\indexsyn{aexp}%
\index{(e1,...,en)@\mbox{$\it \makebox{\tt (}e_1,\ldots ,e_n\makebox{\tt )}$} (tuple)}
%
{\em Tuples}\index{tuple} are written \mbox{$\it \makebox{\tt (}e_1\makebox{\tt ,}\ \ldots \makebox{\tt ,}\ e_k\makebox{\tt )}$}, and may be
of arbitrary length \mbox{$\it k\geq 2$}.  Standard operations on tuples are given
in the standard prelude (see Appendix~\ref{stdprelude}).

\outline{
\paragraph*{Translation:}  
\mbox{$\it \makebox{\tt (}e_1\makebox{\tt ,}\ \ldots \makebox{\tt ,}\ e_k\makebox{\tt )}$} for \mbox{$\it k\geq2$} is an instance of a \mbox{$\it k$}-tuple as
defined in the standard prelude, and requires no translation.  If
\mbox{$\it t_1$} through \mbox{$\it t_k$} are the types of \mbox{$\it e_1$} through \mbox{$\it e_k$},
respectively, then the type of the resulting tuple is 
\mbox{$\it \makebox{\tt (}t_1\makebox{\tt ,}\ \ldots \makebox{\tt ,}\ t_k\makebox{\tt )}$} (see Section~\ref{type-syntax}).
}

\subsection{Unit Expressions and Parenthesised Expressions}
\label{unit-expression}
\index{unit expression}
%
\begin{flushleft}\it\begin{tabbing}
\hspace{0.5in}\=\hspace{3.0in}\=\kill
$\it aexp$\>\makebox[3.5em]{$\rightarrow$}$\it \makebox{\tt ()}$\\ 
$\it $\>\makebox[3.5em]{$|$}$\it \makebox{\tt (}\ exp\ \makebox{\tt )}$
\end{tabbing}\end{flushleft}
\indexsyn{aexp}%
%
\noindent
The form \mbox{$\it \makebox{\tt (}e\makebox{\tt )}$} is simply a {\em parenthesised expression}, and is
equivalent to \mbox{$\it e$}.  The {\em unit expression} \mbox{\tt ()} has type
\mbox{\tt ()}\index{trivial type} (see
Section~\ref{type-syntax}); it is the only member of that type (it can
be thought of as the ``nullary tuple'')---see Section~\ref{basic-trivial}.
\nopagebreak[4]
\outline{
\paragraph{Translation:}  
\mbox{$\it \makebox{\tt (}e\makebox{\tt )}$} is equivalent to \mbox{$\it e$}.
}

\subsection{Arithmetic Sequences}
\label{arithmetic-sequences}
%
\begin{flushleft}\it\begin{tabbing}
\hspace{0.5in}\=\hspace{3.0in}\=\kill
$\it aexp$\>\makebox[3.5em]{$\rightarrow$}$\it \makebox{\tt [}\ exp_1\ [\makebox{\tt ,}\ exp_2]\ \makebox{\tt ..}\ [exp_3]\ \makebox{\tt ]}$
\end{tabbing}\end{flushleft}
\indexsyn{aexp}%
\noindent
The form \mbox{$\it \makebox{\tt [}e_1\makebox{\tt ,}\ e_2\ \makebox{\tt ..}\ e_3\makebox{\tt ]}$} denotes an {\em arithmetic
sequence}\index{arithmetic sequence} from \mbox{$\it e_1$} in increments of
\mbox{$\it e_2-e_1$} up to \mbox{$\it e_3$} (if the increment is positive) or down to \mbox{$\it e_3$}
(if the increment is negative).  An infinite list of \mbox{$\it e_1$}'s results
if the increment is zero, and the empty list results if \mbox{$\it e_3$} is less
than \mbox{$\it e_1$} and the increment is positive, or if \mbox{$\it e_3$} is greater than
\mbox{$\it e_1$} and the increment is negative.  If the comma and \mbox{$\it e_2$} are
omitted, then the increment is 1; if \mbox{$\it e_3$} is omitted, then
the sequence is includes all elements of the enumeration, and is thus
infinite for infinite enumerations.

Arithmetic sequences may be defined over any type in class \mbox{\tt Enum},
including \mbox{\tt Char}, \mbox{\tt Int}, and \mbox{\tt Integer} (see
Figure~\ref{standard-classes} and Section~\ref{derived-decls}).
For example, \mbox{\tt ['a'..'z']} denotes
the list of lower-case letters in alphabetical order.

\outline{
\paragraph{Translation:}
Arithmetic sequences satisfy these identities:
\begin{center}
\begin{tabular}{lcl}%
\struthack{17pt}%
\mbox{\tt [\ }$e_1$\mbox{\tt ..\ ]}         & $=$ 
                        & \mbox{\tt enumFrom} $e_1$ \\
\mbox{\tt [\ }$e_1$\mbox{\tt ,}$e_2$\mbox{\tt ..\ ]} & $=$ 
                        & \mbox{\tt enumFromThen} $e_1$ $e_2$ \\
\mbox{\tt [\ }$e_1$\mbox{\tt ..}$e_3$\mbox{\tt \ ]}  & $=$ 
                        & \mbox{\tt enumFromTo} $e_1$ $e_3$ \\
\mbox{\tt [\ }$e_1$\mbox{\tt ,}$e_2$\mbox{\tt ..}$e_3$\mbox{\tt \ ]} 
                        & $=$ 
                        & \mbox{\tt enumFromThenTo} $e_1$ $e_2$ $e_3$
\end{tabular}
\end{center}
where \mbox{\tt enumFrom}, \mbox{\tt enumFromThen}, \mbox{\tt enumFromTo}, and \mbox{\tt enumFromThenTo}
are operations in the class \mbox{\tt Enum} as defined in the standard prelude
(see Figure~\ref{standard-classes}).
}

\subsection{List Comprehensions}
\index{list comprehension}
\label{list-comprehensions}
%
\begin{flushleft}\it\begin{tabbing}
\hspace{0.5in}\=\hspace{3.0in}\=\kill
$\it aexp$\>\makebox[3.5em]{$\rightarrow$}$\it \makebox{\tt [}\ exp\ \makebox{\tt |}\ qual_1\ \makebox{\tt ,}\ \ldots \ \makebox{\tt ,}\ qual_n\ \makebox{\tt ]}$\>\makebox[3em]{}$\it (\tr{list\ comprehension},\ n\geq 1)$\\ 
$\it qual$\>\makebox[3.5em]{$\rightarrow$}$\it pat\ \makebox{\tt <-}\ exp$\\ 
$\it $\>\makebox[3.5em]{$|$}$\it exp$
\end{tabbing}\end{flushleft}
\indexsyn{aexp}
\indexsyn{qual}

\noindent
A {\em list comprehension} has the form \mbox{$\it \makebox{\tt [}\ e\ \makebox{\tt |}\ q_1\makebox{\tt ,}\ \ldots \makebox{\tt ,}\ q_n\ \makebox{\tt ]},\\
\it n\geq 1,$} where the \mbox{$\it q_i$} qualifiers\index{qualifier} are either {\em
generators}\index{generator} of the form \mbox{$\it p\ \makebox{\tt <-}\ e$}, where \mbox{$\it p$} is a
pattern (see Section~\ref{pattern-matching}) of type \mbox{$\it t$} and \mbox{$\it e$} is an
expression of type \mbox{$\it \makebox{\tt [}t\makebox{\tt ]}$}; or {\em guards},\index{guard} which are
arbitrary expressions of type \mbox{\tt Bool}.

Such a list comprehension returns the list of elements
produced by evaluating \mbox{$\it e$} in the successive environments
created by the nested, depth-first evaluation of the generators in the
qualifier list.  Binding of variables occurs according to the normal
pattern-matching rules (see Section~\ref{pattern-matching}), and if a
match fails then that element of the list is simply skipped over.  Thus:\nopagebreak[4]
\bprog
\mbox{\tt [\ x\ |\ \ xs\ \ \ <-\ [\ [(1,2),(3,4)],\ [(5,4),(3,2)]\ ],\ }\\
\mbox{\tt \ \ \ \ \ \ (3,x)\ <-\ xs\ ]}
\eprog
yields the list \mbox{\tt [4,2]}.  If a qualifier is a guard, it must evaluate
to \mbox{\tt True} for the previous pattern-match to succeed.  
As usual, bindings in list comprehensions can shadow those in outer
scopes; for example:
\[\ba{lll}
\mbox{\tt [\ x\ |\ x\ <-\ x,\ x\ <-\ x\ ]} & = & \mbox{\tt [\ z\ |\ y\ <-\ x,\ z\ <-\ y]} \\
\ea\]
\outline{
\paragraph{Translation:}
List comprehensions satisfy these identities, which may be
used as a translation into the kernel:
\begin{center}
\bt{lcl}%
\struthack{17pt}%
\mbox{\tt [\ }$e$\mbox{\tt \ |\ }$b$\mbox{\tt \ ]}            &=& \mbox{\tt if\ }$b$\mbox{\tt \ then\ [}$e$\mbox{\tt ]\ else\ []} \\
\mbox{\tt [\ }$e$\mbox{\tt \ |\ }$q_1$\mbox{\tt ,\ }$q_2$\mbox{\tt \ ]} &=& \mbox{\tt concat\ [\ [\ }$e$\mbox{\tt \ |\ }$q_2$\mbox{\tt \ ]\ |\ }$q_1$\mbox{\tt \ ]} \\
\mbox{\tt [\ }$e$\mbox{\tt \ |\ }$p$\mbox{\tt \ <-\ }$l$\mbox{\tt \ ]}   &=& \mbox{\tt let\ ok\ }$p$\mbox{\tt \ \ =\ \ True} \\
                               & & \mbox{\tt \ \ \ \ ok\ {\char'137}\ \ =\ \ False} \\
                               & & \mbox{\tt in} \\
                               & & \mbox{\tt \ map\ ({\char'134}}$p$\mbox{\tt \ ->\ }$e$\mbox{\tt )\ (filter\ ok\ }$l$\mbox{\tt )} \\
\et
\end{center}
where \mbox{$\it e$} ranges over expressions, \mbox{$\it p$} ranges over
patterns, \mbox{$\it l$} ranges over list-valued expressions, \mbox{$\it b$} ranges over
boolean expressions, \mbox{$\it q_1$} and \mbox{$\it q_2$} range over non-empty lists of
qualifiers, and \mbox{\tt ok} is a new identifier not appearing in \mbox{$\it e$}, \mbox{$\it p$}, or
\mbox{$\it l$}.  These three equations uniquely define list comprehensions.
\mbox{\tt True}, \mbox{\tt False}, \mbox{\tt map}, \mbox{\tt concat} and \mbox{\tt filter} are all as defined in the
standard prelude.
}

\subsection{Let Expressions}
\index{let expression}
\label{let-expressions}
%
% Including this syntax blurb does REALLY bad things to page breaking
% in the 1.1 report (sigh); ToDo: hope it goes away.
%@@
%exp    ->  \mbox{\tt let\ {\char'173}} decls \mbox{\tt {\char'175}\ in} exp
%@@
\indexsyn{exp}
\index{declaration!within a {\ptt let} expression}
\noindent
{\em Let expressions} have the general form
\mbox{$\it \makebox{\tt let\ {\char'173}}\ d_1\ \makebox{\tt ;}\ \ldots \ \makebox{\tt ;}\ d_n\ \makebox{\tt {\char'175}\ in}\ e$},
and introduce a
nested, lexically-scoped, {\em
mutually-recursive} list of declarations (\mbox{\tt let} is often called \mbox{\tt letrec} in
other languages).  The scope of the declarations is the expression \mbox{$\it e$}
and the right hand side of the declarations.  Declarations are
described in Section~\ref{declarations}.  Pattern bindings are matched
lazily as irrefutable patterns.\index{irrefutable pattern}

\outline{
\paragraph*{Translation:} The dynamic semantics of the expression 
\mbox{$\it \makebox{\tt let\ {\char'173}}\ d_1\ \makebox{\tt ;}\ \ldots \ \makebox{\tt ;}\ d_n\ \makebox{\tt {\char'175}\ in}\ e_0$} is captured by this
translation: After removing all type signatures, each
declaration \mbox{$\it d_i$} is translated into an equation of the form 
\mbox{$\it p_i\ \makebox{\tt =}\ e_i$}, where \mbox{$\it p_i$} and \mbox{$\it e_i$} are patterns and expressions
respectively, using the translation in
Section~\ref{function-bindings}.  Once done, these identities
hold, which may be used as a translation into the kernel:
\begin{center}
\bt{lcl}%
\struthack{17pt}%
\mbox{\tt let\ {\char'173}}$p_1$\mbox{\tt \ =\ }$e_1$\mbox{\tt ;\ }...\mbox{\tt ;\ }$p_n$\mbox{\tt \ =\ }$e_n$\mbox{\tt {\char'175}\ in} $e_0$
      &=& \mbox{\tt let\ ({\char'176}}$p_1$\mbox{\tt ,}...\mbox{\tt ,{\char'176}}$p_n$\mbox{\tt )\ =\ (}$e_1$\mbox{\tt ,}...\mbox{\tt ,}$e_n$\mbox{\tt )\ in} $e_0$ \\
\mbox{\tt let\ }$p$\mbox{\tt \ =\ }$e_1$ \mbox{\tt \ in\ } $e_0$
        &=& \mbox{\tt case\ }$e_1$\mbox{\tt \ of\ {\char'176}}$p$\mbox{\tt \ ->\ }$e_0$   \\
        & & {\rm where no variable in $p$ appears free in $e_1$} \\
\mbox{\tt let\ }$p$\mbox{\tt \ =\ }$e_1$ \mbox{\tt \ in\ } $e_0$
      &=& \mbox{\tt let\ }$p$\mbox{\tt \ =\ fix\ (\ {\char'134}\ {\char'176}}$p$\mbox{\tt \ ->\ }$e_1$\mbox{\tt )\ in} $e_0$
\et
\end{center}
where \mbox{\tt fix} is the least fixpoint operator.  Note the use of the
irrefutable patterns in the second and third rules.  
% This same semantics applies to the top-level of
%a program that has been translated into a \mbox{\tt let} expression,
% as described at the beginning of Section~\ref{modules}.
The static semantics of the bindings in a \mbox{\tt let} expression
is described in 
Section~\ref{pattern-bindings}.
}

\subsection{Case Expressions}
\label{case}
%
\begin{flushleft}\it\begin{tabbing}
\hspace{0.5in}\=\hspace{3.0in}\=\kill
$\it exp$\>\makebox[3.5em]{$\rightarrow$}$\it \makebox{\tt case}\ exp\ \makebox{\tt of}\ \makebox{\tt {\char'173}}\ alts\ [\makebox{\tt ;}]\ \makebox{\tt {\char'175}}$\\ 
$\it alts$\>\makebox[3.5em]{$\rightarrow$}$\it alt_1\ \makebox{\tt ;}\ \ldots \ \makebox{\tt ;}\ alt_n$\>\makebox[3em]{}$\it (n\geq 0)$\\ 
$\it alt$\>\makebox[3.5em]{$\rightarrow$}$\it pat\ \makebox{\tt ->}\ exp\ [\makebox{\tt where}\ \makebox{\tt {\char'173}}\ decls\ [\makebox{\tt ;}]\ \makebox{\tt {\char'175}}]$\\ 
$\it $\>\makebox[3.5em]{$|$}$\it pat\ gdpat\ [\makebox{\tt where}\ \makebox{\tt {\char'173}}\ decls\ [\makebox{\tt ;}]\ \makebox{\tt {\char'175}}]$\\ 
$\it $\\ 
$\it gdpat$\>\makebox[3.5em]{$\rightarrow$}$\it gd\ \makebox{\tt ->}\ exp\ [\ gdpat\ ]$\\ 
$\it gd$\>\makebox[3.5em]{$\rightarrow$}$\it \makebox{\tt |}\ exp$
\end{tabbing}\end{flushleft}
\indexsyn{exp}%
\indexsyn{alts}%
\indexsyn{alt}%
\indexsyn{gdpat}%
\indexsyn{gd}%
%\indextt{case ... of ...}
A {\em case expression}\index{case expression} has the general form
\[
\mbox{$\it \makebox{\tt case}\ e\ \makebox{\tt of\ {\char'173}\ }p_1\ match_1\ \makebox{\tt ;}\ \ldots \ \makebox{\tt ;}\ p_n\ match_n\ \makebox{\tt {\char'175}}$}
\]
where each \mbox{$\it match_i$} is of the general form
\[\ba{lll}
 & \mbox{$\it \makebox{\tt |}\ g_{i1}$}   & \mbox{$\it \makebox{\tt ->}\ e_{i1}\ \makebox{\tt ;}$} \\
 & \mbox{$\it \ldots $} \\
 & \mbox{$\it \makebox{\tt |}\ g_{im_i}$} & \mbox{$\it \makebox{\tt ->}\ e_{im_i}$} \\
 & \multicolumn{2}{l}{\mbox{$\it \makebox{\tt where\ {\char'173}}\ decls_i\ \makebox{\tt {\char'175}}$}}
\ea\]
where each clause \mbox{$\it p_i\ matches_i$} consists of a a {\em
pattern}\index{pattern} \mbox{$\it p_i$} and its \mbox{$\it matches_i$}, which
consists of pairs of optional {\em guards}\index{guard}
\mbox{$\it g_{ij}$} and {\em bodies} \mbox{$\it e_{ij}$} (expressions), as well as
optional bindings (\mbox{$\it decls_i$}) that scope over all of the guards and
expressions of the clause.  An alternative of the form
\[
\mbox{$\it pat\ \makebox{\tt ->}\ expr\ \makebox{\tt where\ {\char'173}}\ decls\ \makebox{\tt {\char'175}}$}
\]
is treated as shorthand for:
\[\ba{lll}
 & \mbox{$\it pat\ \makebox{\tt |\ True}$}   & \mbox{$\it \makebox{\tt ->}\ expr$} \\
 & \multicolumn{2}{l}{\mbox{$\it \makebox{\tt where\ {\char'173}}\ decls\ \makebox{\tt {\char'175}}$}}
\ea\]

A case expression must have at least one clause and each clause must
have at least one body.  Each body must have the same type, and the
type of the whole expression is that type.

A case expression is evaluated by pattern-matching the expression \mbox{$\it e$}
against the individual clauses.  The matches are tried sequentially,
from top to bottom.  The first successful match causes evaluation of
the corresponding clause body, in the environment of the case
expression extended by the bindings created during the matching of
that clause and by the \mbox{$\it decls_i$} associated with that clause.  If no
match succeeds, the result is \mbox{$\it \bot$}.  Pattern matching is described
in Section~\ref{pattern-matching}, with the formal semantics of case
expressions in Section~\ref{case-semantics}.

\subsection{Expression Type-Signatures}
\index{expression type-signature}
\label{expression-type-sigs}
%
\begin{flushleft}\it\begin{tabbing}
\hspace{0.5in}\=\hspace{3.0in}\=\kill
$\it exp$\>\makebox[3.5em]{$\rightarrow$}$\it exp\ \makebox{\tt ::}\ [context\ \makebox{\tt =>}]\ atype$
\end{tabbing}\end{flushleft}
\indexsyn{exp}
\indextt{::}
\nopagebreak[4]
{\em Expression type-signatures} have the form \mbox{$\it e\ \makebox{\tt ::}\ t$}, where \mbox{$\it e$}
is an expression and \mbox{$\it t$} is a type (Section~\ref{type-syntax}); they
are used to type an expression explicitly
and may be used to resolve ambiguous typings due to overloading (see
Section~\ref{default-decls}).  The value of the expression is just that of
\mbox{$\it exp$}.  As with normal type signatures (see
Section~\ref{type-signatures}), the declared type may be more specific than 
the principal type derivable from \mbox{$\it exp$}, but it is an error to give
a type that is more general than, or not comparable to, the
principal type.


\subsection{Pattern-Matching}
\index{pattern-matching}
\label{pattern-matching}
\label{patterns}

{\em Patterns} appear in lambda abstractions, function definitions, pattern
bindings, list comprehensions, and case expressions.  However, the
first four of these ultimately translate into case expressions, so
defining the semantics of pattern-matching for case expressions is sufficient.
%it suffices to restrict the definition of the semantics of
%pattern-matching to case expressions.

\subsubsection{Patterns}

Patterns\index{pattern} have this syntax:
\begin{flushleft}\it\begin{tabbing}
\hspace{0.5in}\=\hspace{3.0in}\=\kill
$\it pat$\>\makebox[3.5em]{$\rightarrow$}$\it pat^0$\\ 
$\it pat^i$\>\makebox[3.5em]{$\rightarrow$}$\it pat^{i+1}_1\ [conop^{({\rm\ n},i)}\ pat^{i+1}_2]$\>\makebox[3em]{}$\it (0\leq i\leq 9)$\\ 
$\it $\>\makebox[3.5em]{$|$}$\it lpat^i\ conop^{({\rm\ l},i)}\ pat^{i+1}$\\ 
$\it $\>\makebox[3.5em]{$|$}$\it pat^{i+1}\ conop^{({\rm\ r},i)}\ rpat^i$\\ 
$\it lpat^i$\>\makebox[3.5em]{$\rightarrow$}$\it [lpat^i\ conop^{({\rm\ l},i)}]\ pat^{i+1}$\>\makebox[3em]{}$\it (0\leq i\leq 9)$\\ 
$\it lpat^6$\>\makebox[3.5em]{$\rightarrow$}$\it lpat^6\ \makebox{\tt +}\ integer$\>\makebox[3em]{}$\it (\tr{successor\ pattern})$\\ 
$\it $\>\makebox[3.5em]{$|$}$\it \makebox{\tt -}\ \{integer\ |\ float\}$\>\makebox[3em]{}$\it (\tr{negative\ literal})$\\ 
$\it rpat^i$\>\makebox[3.5em]{$\rightarrow$}$\it pat^{i+1}\ [conop^{({\rm\ r},i)}\ rpat^i]$\>\makebox[3em]{}$\it (0\leq i\leq 9)$\\ 
$\it pat^{10}$\>\makebox[3.5em]{$\rightarrow$}$\it apat$\\ 
$\it $\>\makebox[3.5em]{$|$}$\it con\ apat_1\ \ldots \ apat_k$\>\makebox[3em]{}$\it (\arity{con}=k\geq 1)$\\ 
$\it $\\ 
$\it apat$\>\makebox[3.5em]{$\rightarrow$}$\it var\ [{\tt\ @}\ apat]$\>\makebox[3em]{}$\it (\tr{as\ pattern})$\\ 
$\it $\>\makebox[3.5em]{$|$}$\it con$\>\makebox[3em]{}$\it (\arity{con}=0)$\\ 
$\it $\>\makebox[3.5em]{$|$}$\it literal$\\ 
$\it $\>\makebox[3.5em]{$|$}$\it \makebox{\tt {\char'137}}$\>\makebox[3em]{}$\it (\tr{wildcard})$\\ 
$\it $\>\makebox[3.5em]{$|$}$\it \makebox{\tt ()}$\>\makebox[3em]{}$\it (\tr{unit\ pattern})$\\ 
$\it $\>\makebox[3.5em]{$|$}$\it \makebox{\tt (}\ pat\ \makebox{\tt )}$\>\makebox[3em]{}$\it (\tr{parenthesised\ pattern})$\\ 
$\it $\>\makebox[3.5em]{$|$}$\it \makebox{\tt (}\ pat_1\ \makebox{\tt ,}\ \ldots \ \makebox{\tt ,}\ pat_k\ \makebox{\tt )}$\>\makebox[3em]{}$\it (\tr{tuple\ pattern},\ k\geq 2)$\\ 
$\it $\>\makebox[3.5em]{$|$}$\it \makebox{\tt [}\ pat_1\ \makebox{\tt ,}\ \ldots \ \makebox{\tt ,}\ pat_k\ \makebox{\tt ]}$\>\makebox[3em]{}$\it (\tr{list\ pattern},\ k\geq 0)$\\ 
$\it $\>\makebox[3.5em]{$|$}$\it \makebox{\tt {\char'176}}\ apat$\>\makebox[3em]{}$\it (\tr{irrefutable\ pattern})$
\end{tabbing}\end{flushleft}
\indexsyn{pat}%
\index{pat@\mbox{$\it pat^i$}}%
\index{lpat@\mbox{$\it lpat^i$}}%
\index{rpat@\mbox{$\it rpat^i$}}%
\indexsyn{apat}%
The arity of a constructor must match the number of
sub-patterns associated with it; one cannot match against a
partially-applied constructor.

All patterns must be {\em linear}\index{linearity}\index{linear
pattern}---no variable may appear more than once.

Patterns of the form \mbox{$\it var$}{\tt @}\mbox{$\it pat$} are called {\em as-patterns},
\index{as-pattern ({\ptt {\char'100}})}
and allow one to use \mbox{$\it var$}
as a name for the value being matched by \mbox{$\it pat$}.  For example,\nopagebreak[4]
\bprog
\mbox{\tt case\ e\ of\ {\char'173}\ xs@(x:rest)\ ->\ if\ x==0\ then\ rest\ else\ xs\ {\char'175}}
\eprog
is equivalent to:
\bprog
\mbox{\tt let\ {\char'173}\ xs\ =\ e\ {\char'175}\ in}\\
\mbox{\tt \ \ case\ xs\ of\ {\char'173}\ (x:rest)\ ->\ if\ x\ ==\ 0\ then\ rest\ else\ xs\ {\char'175}}
\eprog
%This transformation of a case expression is always valid, and
%is assumed done prior to the pattern-matching semantics given below.

Patterns of the form \mbox{\tt {\char'137}} are {\em
wildcards}\index{wildcard pattern ({\ptt {\char'137}})} and are useful when some part of a pattern
is not referenced on the right-hand-side.  It is as if an
identifier not used elsewhere were put in its place.  For example,
\bprog
\mbox{\tt case\ e\ of\ {\char'173}\ [x,{\char'137},{\char'137}]\ \ ->\ \ if\ x==0\ then\ True\ else\ False\ {\char'175}}
\eprog
is equivalent to:
\bprog
\mbox{\tt case\ e\ of\ {\char'173}\ [x,y,z]\ \ ->\ \ if\ x==0\ then\ True\ else\ False\ {\char'175}}
\eprog
% where \mbox{\tt y} and \mbox{\tt z} are identifiers not used elsewhere.

%old:
%This translation is also 
%assumed prior to the semantics given below.

In the pattern-matching rules given below we distinguish two kinds of
patterns: an {\em irrefutable pattern}
\index{irrefutable pattern}
is either of the form \mbox{$\it \makebox{\tt {\char'176}}apat$}, a variable, or a wildcard; all
other patterns are {\em refutable}.
\index{refutable pattern}

\subsubsection{Informal semantics of pattern-matching}

Patterns are matched against values.  Attempting to match a pattern
can have one of three results: it may {\em fail\/}; it may {\em
succeed}, returning a binding for each variable in the pattern; or it
may {\em diverge} (i.e.~return \mbox{$\it \bot$}).  Pattern-matching proceeds
from left to right, and outside in, according to these rules:
\begin{enumerate}
\item Matching a value \mbox{$\it v$} against the irrefutable pattern
\index{irrefutable pattern}
\mbox{$\it var$} always succeeds and binds \mbox{$\it var$} to \mbox{$\it v$}.  Similarly, matching \mbox{$\it v$}
against the irrefutable pattern \mbox{$\it \makebox{\tt {\char'176}}apat$} always succeeds.  The free
variables in \mbox{$\it apat$} are bound to the appropriate values if matching
\mbox{$\it v$} against \mbox{$\it apat$} would otherwise succeed, and to \mbox{$\it \bot$} if matching
\mbox{$\it v$} against \mbox{$\it apat$} fails or diverges.  (Binding does {\em
not} imply evaluation.)

Operationally, this means that no matching is done on an
irrefutable pattern until one of the variables in the pattern is used.
At that point the entire pattern is matched against the value, and if
the match fails or diverges, so does the overall computation.

\item Matching \mbox{$\it \bot$} against a refutable pattern always diverges.
\index{refutable pattern}

\item Matching a non-\mbox{$\it \bot$} value can occur against two kinds of
refutable patterns:
\begin{enumerate}
\item Matching a non-\mbox{$\it \bot$} value against a constructed pattern
\index{constructed pattern}
fails if the outermost constructors are different.  If the constructors are
the same, the result of the match is the result of matching the
sub-patterns left-to-right: if all matches succeed, the overall match
succeeds; the first to fail or diverge causes the overall match to
fail or diverge, respectively.  

Constructed values consist of those created by prefix or infix
constructors, tuple or list patterns, and strings (which are
lists of characters).  Characters and \mbox{\tt ()} are treated as
nullary constructors.  Numeric literals are matched using the
overloaded \mbox{\tt ==} function.

%Also, literals (characters, positive and
%negative integers, and the unit value \mbox{\tt ()}) are treated as nullary
%constructors.

\item Matching a non-\mbox{$\it \bot$} value \mbox{$\it n$} against a pattern of the form
\mbox{$\it x\makebox{\tt +}k$}
\index{n+k pattern@\mbox{$\it n\makebox{\tt +}k$} pattern}
(where \mbox{$\it x$} is a variable and \mbox{$\it k$} is a positive integer
literal) succeeds if \mbox{$\it n\geq k$}, resulting in the binding of \mbox{$\it x$} to \mbox{$\it n-k$},
and fails if \mbox{$\it n<k$}.  For example, the Fibonacci function may be
defined as follows:

\bprog
\mbox{\tt fib\ n\ =\ case\ n\ of\ {\char'173}}\\
\mbox{\tt \ \ \ \ \ \ \ \ \ \ 0\ \ \ ->\ 1\ ;}\\
\mbox{\tt \ \ \ \ \ \ \ \ \ \ 1\ \ \ ->\ 1\ ;}\\
\mbox{\tt \ \ \ \ \ \ \ \ \ \ n+2\ ->\ fib\ n\ +\ fib\ (n+1)\ {\char'175}}
\eprog
Since \mbox{\tt n} must be bound to a positive value, \mbox{\tt fib} diverges for a
negative argument, and exactly one of the equations matches any
non-negative argument.
\end{enumerate}

\item
The result of matching a value \mbox{$\it v$} against an as-pattern \mbox{$\it var$}{\tt @}\mbox{$\it pat$} is
\index{as-pattern ({\ptt {\char'100}})}
the result of matching \mbox{$\it v$} against \mbox{$\it pat$} augmented with the binding of
\mbox{$\it var$} to \mbox{$\it v$}.  If the match of \mbox{$\it v$} against \mbox{$\it pat$} fails or diverges,
then so does the overall match.
\end{enumerate}

Aside from the obvious static type constraints (for
example, it is a static error to match a character against a
boolean), these static class constraints hold: an integer
literal pattern
\index{integer literal pattern}
can only be matched against a value in the class
\mbox{\tt Num}; a floating literal pattern
\index{floating literal pattern}
can only be matched against a value
in the class \mbox{\tt Fractional}; and a \mbox{$\it n\makebox{\tt +}k$} pattern
\index{n+k pattern@\mbox{$\it n\makebox{\tt +}k$} pattern}
can only be matched
against a value in the class \mbox{\tt Integral}.

Here are some examples:
\begin{enumerate}
\item If the pattern \mbox{\tt [1,2]} is matched against \mbox{$\it \makebox{\tt [0,}\bot\makebox{\tt ]}$}, then \mbox{\tt 1}
\mbox{$\it fails$} to match against \mbox{\tt 0}, and the result is a failed match.  But
if \mbox{\tt [1,2]} is matched against \mbox{$\it \makebox{\tt [}\bot\makebox{\tt ,0]}$}, then attempting to match
\mbox{\tt 1} against \mbox{$\it \bot$} causes the match to \mbox{$\it diverge$}.

\item These examples demonstrate refutable vs.~irrefutable
matching:
\nopagebreak[4]
\bprog
\mbox{\tt ({\char'134}\ {\char'176}(x,y)\ ->\ 0)\ }\mbox{$\it \bot$}\mbox{\tt \ \ \ \ }\mbox{$\it \Rightarrow$}\mbox{\tt \ \ \ \ 0}\\
\mbox{\tt ({\char'134}\ \ (x,y)\ ->\ 0)\ }\mbox{$\it \bot$}\mbox{\tt \ \ \ \ }\mbox{$\it \Rightarrow$}\mbox{\tt \ \ \ \ }\mbox{$\it \bot$}\mbox{\tt }
\eprog
\bprog
\mbox{\tt ({\char'134}\ {\char'176}[x]\ ->\ 0)\ []\ \ \ \ }\mbox{$\it \Rightarrow$}\mbox{\tt \ \ \ \ 0}\\
\mbox{\tt ({\char'134}\ {\char'176}[x]\ ->\ x)\ []\ \ \ \ }\mbox{$\it \Rightarrow$}\mbox{\tt \ \ \ \ }\mbox{$\it \bot$}\mbox{\tt }
\eprog
\bprog
\mbox{\tt ({\char'134}\ {\char'176}[x,{\char'176}(a,b)]\ ->\ x)\ [(0,1),}\mbox{$\it \bot$}\mbox{\tt ]\ \ \ \ }\mbox{$\it \Rightarrow$}\mbox{\tt \ \ \ \ (0,1)}\\
\mbox{\tt ({\char'134}\ {\char'176}[x,\ (a,b)]\ ->\ x)\ [(0,1),}\mbox{$\it \bot$}\mbox{\tt ]\ \ \ \ }\mbox{$\it \Rightarrow$}\mbox{\tt \ \ \ \ }\mbox{$\it \bot$}\mbox{\tt }
\eprog
\bprog
\mbox{\tt ({\char'134}\ \ (x:xs)\ ->\ x:x:xs)\ }\mbox{$\it \bot$}\mbox{\tt \ \ \ }\mbox{$\it \Rightarrow$}\mbox{\tt \ \ \ }\mbox{$\it \bot$}\mbox{\tt }\\
\mbox{\tt ({\char'134}\ {\char'176}(x:xs)\ ->\ x:x:xs)\ }\mbox{$\it \bot$}\mbox{\tt \ \ \ }\mbox{$\it \Rightarrow$}\mbox{\tt \ \ \ }\mbox{$\it \bot$}\mbox{\tt :}\mbox{$\it \bot$}\mbox{\tt :}\mbox{$\it \bot$}\mbox{\tt }
\eprogNoSkip
\end{enumerate}

Top level patterns in case
expressions, and the set of top level patterns in function or pattern
bindings, may have zero or more associated {\em guards}\index{guard}.  A guard is
a boolean expression that is evaluated only after all of the
arguments have been successfully matched, and it must be true for the
overall pattern-match to succeed.  The environment of the guard is the same
as the right-hand-side of the case expression
clause, function definition, or pattern binding to which it is attached.

The guard semantics has an obvious influence on the
strictness characteristics of a function or case expression.  In
particular, an otherwise irrefutable pattern
\index{irrefutable pattern}
may be evaluated because of a guard.  For example, in
\bprog
\mbox{\tt f\ {\char'176}(x,y,z)\ [a]\ |\ a==y\ =\ 1}
\eprog
both \mbox{\tt a} and \mbox{\tt y} will be evaluated.


\subsubsection{Formal semantics of pattern-matching}
\label{case-semantics}

The semantics of all other constructs that use
pattern-matching is defined by giving identities that relate those constructs to
\mbox{\tt case} expressions.

The semantics of \mbox{\tt case} expressions are given as a series of
identities. Figure~\ref{simple-case-expr} shows the
identities:
$e$, $e'$ and $e_i$ are arbitrary expressions; 
$g$ and $g_i$ are boolean-valued expressions; 
$p$ and $p_i$ are patterns; 
$x$ and $x_i$ are variables; 
$K$ and $K'$ are constructors (including tuple constructors); 
a $match_i$ is a form as shown in rule~(a);
and $k$ is a character, string, or numeric literal.

For clarity, several rules are expressed using
\mbox{\tt let} (used only in a non-recursive way); their usual purpose is to
prevent name capture
(e.g., in rule~(b)).  The rules may be re-expressed entirely with
\mbox{\tt cases} by applying this identity:
\[
\mbox{$\it \makebox{\tt let\ }x\makebox{\tt \ =\ }y\makebox{\tt \ in\ }e\makebox{\tt \ }\ =\makebox{\tt \ \ case\ }y\makebox{\tt \ of\ {\char'173}\ }x\makebox{\tt \ ->\ }e\makebox{\tt \ {\char'175}}$}
\]

\begin{figure}
\outline{
\begin{tabular}{@{}cl}
(a)&\mbox{\tt case\ }$e_0$\mbox{\tt \ of\ {\char'173}\ }$p_1\ \ match_1$\mbox{\tt ;\ \ }$\ldots{}$\mbox{\tt \ ;\ }$p_n\ \ match_n$\mbox{\tt \ {\char'175}}\\
   &$=$\mbox{\tt \ \ case\ }$e_0$\mbox{\tt \ of\ {\char'173}\ }$p_1\ \ match_1$\mbox{\tt \ ;}\\
   &\mbox{\tt \ \ \ \ \ \ \ \ \ \ \ \ \ \ \ \ {\char'137}\ \ ->\ }$\ldots{}$\mbox{\tt \ case\ }$e_0$\mbox{\tt \ of\ {\char'173}}\\
   &\mbox{\tt \ \ \ \ \ \ \ \ \ \ \ \ \ \ \ \ \ \ \ \ \ \ \ \ \ \ \ }$p_n\ \ match_n$\\
   &\mbox{\tt \ \ \ \ \ \ \ \ \ \ \ \ \ \ \ \ \ \ \ \ \ \ \ \ \ \ \ {\char'137}\ \ ->\ error\ "No\ match"\ {\char'175}}$\ldots$\mbox{\tt {\char'175}}\\
   &\mbox{\tt \ }{\rm where each $match_i$ has the form:}\\
   &\mbox{\tt \ \ |\ }$g_{i,1}$  \mbox{\tt \ ->\ }$e_{i,1}$\mbox{\tt \ ;\ }$\ldots{}$\mbox{\tt \ ;\ |\ }$g_{i,m_i}$\mbox{\tt \ ->\ }$e_{i,m_i}$\mbox{\tt \ where\ {\char'173}\ }$decls_i$\mbox{\tt \ {\char'175}}\\[4pt]
%\\
(b)&\mbox{\tt case\ }$e_0$\mbox{\tt \ of\ {\char'173}\ }$p$\mbox{\tt \ |\ }$g_1$\mbox{\tt \ ->\ }$e_1$\mbox{\tt \ ;\ }$\ldots{}$\\
   &\hspace*{4pt}\mbox{\tt \ \ \ \ \ \ \ \ \ \ \ \ \ \ |\ }$g_n$\mbox{\tt \ ->\ }$e_n$\mbox{\tt \ where\ {\char'173}\ }$decls$\mbox{\tt \ {\char'175}}\\
   &\hspace*{2pt}\mbox{\tt \ \ \ \ \ \ \ \ \ \ \ \ {\char'137}\ \ \ \ \ \ ->\ }$e'$\mbox{\tt \ {\char'175}}\\
   &$=$\mbox{\tt \ let\ {\char'173}\ }$y$\mbox{\tt \ =\ }$e'$\mbox{\tt \ {\char'175}\ \ \ }(where $y$ is a completely new variable)\\
   &\mbox{\tt \ \ \ in\ \ case\ }$e_0$\mbox{\tt \ of\ {\char'173}}\\
   &\mbox{\tt \ \ \ \ \ \ \ \ \ }$p$\mbox{\tt \ ->\ let\ {\char'173}\ }$decls$\mbox{\tt \ {\char'175}\ in}\\
   &\mbox{\tt \ \ \ \ \ \ \ \ \ \ \ \ \ \ \ \ if\ }$g_1$\mbox{\tt \ then\ }$e_1$\mbox{\tt \ }$\ldots{}$\mbox{\tt \ else\ if\ }$g_n$\mbox{\tt \ then\ }$e_n$\mbox{\tt \ else\ }$y$\\
   &\mbox{\tt \ \ \ \ \ \ \ \ \ {\char'137}\ ->\ }$y$\mbox{\tt \ {\char'175}}\\[4pt]
%\\
(c)&\mbox{\tt case\ }$e_0$\mbox{\tt \ of\ {\char'173}\ {\char'176}}$p$\mbox{\tt \ ->\ }$e$\mbox{\tt ;\ {\char'137}\ ->\ }$e'$\mbox{\tt \ {\char'175}}\\
&$=$\mbox{\tt \ let\ {\char'173}\ }$y$\mbox{\tt \ =\ }$e_0$\mbox{\tt \ {\char'175}\ in}\\
&\mbox{\tt \ \ \ let\ {\char'173}\ }$x'_1$\mbox{\tt \ =\ case\ }$y$\mbox{\tt \ of\ {\char'173}\ }$p$\mbox{\tt \ ->\ }$x_1$\mbox{\tt \ {\char'175}{\char'175}\ in\ }$\ldots$\\
&\mbox{\tt \ \ \ let\ {\char'173}\ }$x'_n$\mbox{\tt \ =\ case\ }$y$\mbox{\tt \ of\ {\char'173}\ }$p$\mbox{\tt \ ->\ }$x_n$\mbox{\tt \ {\char'175}{\char'175}\ in\ }$e\,[x'_1/x_1, \ldots, x'_n/x_n]$\\[2pt]
&{\rm $x_1, \ldots, x_n$ are all the variables in $p\/$; $y, x'_1, \ldots, x'_n$ are completely new variables}\\[4pt]
%\\
(d)&\mbox{\tt case\ }$e_0$\mbox{\tt \ of\ {\char'173}\ }$x${\tt @}$p$\mbox{\tt \ ->\ }$e$\mbox{\tt ;\ {\char'137}\ ->\ }$e'$\mbox{\tt \ {\char'175}}\\
&$=$\mbox{\tt \ let\ {\char'173}\ }$y$\mbox{\tt \ =\ }$e_0$\mbox{\tt \ {\char'175}\ \ \ }(where $y$ is a completely new variable)\\
&\mbox{\tt \ \ \ in\ \ case\ }$y$\mbox{\tt \ of\ {\char'173}\ }$p$\mbox{\tt \ ->\ (\ {\char'134}\ }$x$\mbox{\tt \ ->\ }$e$\mbox{\tt \ )\ }$y$\mbox{\tt \ ;\ {\char'137}\ ->\ }$e'$\mbox{\tt \ {\char'175}}\\[4pt]
%\\
(e)&\mbox{\tt case\ }$e_0$\mbox{\tt \ of\ {\char'173}\ {\char'137}\ ->\ }$e$\mbox{\tt ;\ {\char'137}\ ->\ }$e'$\mbox{\tt \ {\char'175}\ }$=$\mbox{\tt \ }$e$\\[4pt]
%\\
(f)&\mbox{\tt case\ }$e_0$\mbox{\tt \ of\ {\char'173}\ }$K\ p_1 \ldots p_n$\mbox{\tt \ ->\ }$e$\mbox{\tt ;\ {\char'137}\ ->\ }$e'$\mbox{\tt \ {\char'175}}\\
&$=$\mbox{\tt \ let\ {\char'173}\ }$y$\mbox{\tt \ =\ }$e'$\mbox{\tt \ {\char'175}}\\
&\mbox{\tt \ \ \ in\ \ case\ }$e_0$\mbox{\tt \ of\ {\char'173}}\\
&\mbox{\tt \ \ \ \ \ }$K\ x_1 \ldots x_n$\mbox{\tt \ ->\ case\ }$x_1$\mbox{\tt \ of\ {\char'173}}\\
&\mbox{\tt \ \ \ \ \ \ \ \ \ \ \ \ \ \ \ \ \ \ \ \ }$p_1$\mbox{\tt \ ->\ }$\ldots$\mbox{\tt \ case\ }$x_n$\mbox{\tt \ of\ {\char'173}\ }$p_n$\mbox{\tt \ ->\ }$e$\mbox{\tt \ ;\ {\char'137}\ ->\ }$y$\mbox{\tt \ {\char'175}\ }$\ldots$\\
&\mbox{\tt \ \ \ \ \ \ \ \ \ \ \ \ \ \ \ \ \ \ \ \ {\char'137}\ \ ->\ }$y$\mbox{\tt \ {\char'175}}\\
&\mbox{\tt \ \ \ \ \ {\char'137}\ ->\ }$y$\mbox{\tt \ {\char'175}\ }{\rm (where $y, x_1, \ldots, x_n$ are completely new variables)}\\[4pt]
%\\
(g)&\mbox{\tt case\ }$e_0$\mbox{\tt \ of\ {\char'173}\ }$k$\mbox{\tt \ ->\ }$e$\mbox{\tt ;\ {\char'137}\ ->\ }$e'$\mbox{\tt \ {\char'175}\ }$=$\mbox{\tt \ if\ (}$k$\mbox{\tt \ ==\ }$e_0$\mbox{\tt )\ then\ }$e$\mbox{\tt \ else\ }$e'$\\[4pt]
%\\
(h)&\mbox{\tt case\ }$e_0$\mbox{\tt \ of\ {\char'173}\ }$x$\mbox{\tt +}$k$\mbox{\tt \ ->\ }$e$\mbox{\tt ;\ {\char'137}\ ->\ }$e'$\mbox{\tt \ {\char'175}}\\
&$=$\mbox{\tt \ \ if\ (}$e_0$\mbox{\tt \ >=\ }$k$\mbox{\tt )\ then\ let\ {\char'173}\ }$x$\mbox{\tt \ =\ (}$e_0$\mbox{\tt -}$k$\mbox{\tt )\ {\char'175}\ in\ }$e$\mbox{\tt \ else\ }$e'$\\[4pt]
%\\
(i)&\mbox{\tt case\ }$e_0$\mbox{\tt \ of\ {\char'173}\ }$x$\mbox{\tt \ ->\ }$e$\mbox{\tt ;\ {\char'137}\ ->\ }$e'$\mbox{\tt \ {\char'175}\ }$=$\mbox{\tt \ case\ }$e_0$\mbox{\tt \ of\ {\char'173}\ }$x$\mbox{\tt \ ->\ }$e$\mbox{\tt \ {\char'175}}\\[4pt]
%\\
(j)&\mbox{\tt case\ }$e_0$\mbox{\tt \ of\ {\char'173}\ }$x$\mbox{\tt \ ->\ }$e$\mbox{\tt \ {\char'175}\ }$=$\mbox{\tt \ (\ {\char'134}\ }$x$\mbox{\tt \ ->\ }$e$\mbox{\tt \ )\ }$e_0$\\[4pt]
%\\
(k)&\mbox{\tt case\ (}$K'$\mbox{\tt \ }$e_1$\mbox{\tt \ }$\ldots$\mbox{\tt \ }$e_m$\mbox{\tt )\ of\ {\char'173}\ }$K$\mbox{\tt \ }$x_1$\mbox{\tt \ }$\ldots$\mbox{\tt \ }$x_n$\mbox{\tt \ ->\ }$e$\mbox{\tt ;\ {\char'137}\ ->\ }$e'$\mbox{\tt \ {\char'175}\ }$=$\mbox{\tt \ }$e'$\\
&{\rm where $K$ and $K'$ are distinct constructors of arity $n$ and $m$, respectively}\\[4pt]
%\\
(l)&\mbox{\tt case\ (}$K$\mbox{\tt \ }$e_1$\mbox{\tt \ }$\ldots$\mbox{\tt \ }$e_n$\mbox{\tt )\ of\ {\char'173}\ }$K$\mbox{\tt \ }$x_1$\mbox{\tt \ }$\ldots$\mbox{\tt \ }$x_n$\mbox{\tt \ ->\ }$e$\mbox{\tt ;\ {\char'137}\ ->\ }$e'$\mbox{\tt \ {\char'175}}\\
&$=$\mbox{\tt \ \ case\ }$e_1$\mbox{\tt \ of\ {\char'173}\ }$x_1$\mbox{\tt \ ->\ }$\ldots$\mbox{\tt \ \ case\ }$e_n$\mbox{\tt \ of\ {\char'173}\ }$x_n$\mbox{\tt \ ->\ }$e$\mbox{\tt \ {\char'175}}$\ldots$\mbox{\tt {\char'175}}\\
&{\rm where $K$ is a constructor of arity $n$}
\end{tabular}
}
\ecaption{Semantics of Case Expressions}
\label{simple-case-expr}
\end{figure}

Using all but the last two identities (rules~(k) and~(l)) in Figure~\ref{simple-case-expr}
in a left-to-right manner yields a translation into a
subset of general \mbox{\tt case} expressions called {\em simple case expressions}.%
\index{simple case expression}
Rule~(a) matches a general source-language
\mbox{\tt case} expression, regardless of whether it actually includes
guards---if no guards are written, then \mbox{\tt True} is substituted for the guards $g_{i,j}$
in the $match_i$ forms.
Subsequent identities manipulate the resulting \mbox{\tt case} expression into simpler
and simpler forms.
The semantics of simple \mbox{\tt case} expressions is 
given by the last two identities ((k) and~(l)).

Rules~(g) and~(h) in Figure~\ref{simple-case-expr} involve the
overloaded operators \mbox{\tt ==} and \mbox{\tt <=}; it is these rules that define the
meaning of pattern-matching against overloaded constants.
\index{pattern-matching!overloaded constant}

When used as a translation, the identities in
Figure~\ref{simple-case-expr} will generate a very inefficient
program.  This can be fixed by using further \mbox{\tt case} or \mbox{\tt let} 
expressions, but doing so 
would clutter the identities, which are intended only to convey the semantics.

These identities all preserve the static semantics.  Rules~(d) and~(j)
use a lambda rather than a \mbox{\tt let}; this indicates that variables bound
by \mbox{\tt case} are monomorphically typed (Section~\ref{type-semantics}).
\index{monomorphic type variable}

% Local Variables: 
% mode: latex
% End:
