%
% $Header$
%
\begin{center}
\Large\bf Preface to Version~1.0 (revised)
\end{center}

\vspace{.2in}

\begin{quote}
``{\em Some half dozen persons have written technically on combinatory
logic, and most of these, including ourselves, have published
something erroneous.  Since some of our fellow sinners are among the
most careful and competent logicians on the contemporary scene, we
regard this as evidence that the subject is refractory.  Thus fullness
of exposition is necessary for accuracy; and excessive condensation
would be false economy here, even more than it is ordinarily.}''
\begin{flushright}
Haskell B.~Curry and Robert Feys \\
in the Preface to {\em Combinatory Logic} \cite{curry&feys:book}, May 31, 1956
\end{flushright}
\end{quote}

\vspace{.2in}

\noindent
In September of 1987 a meeting was held at the conference on
Functional Programming
Languages and Computer Architecture in
Portland, Oregon, to discuss an unfortunate situation
in the functional programming community: there had come into being
more than a dozen non-strict, purely functional programming languages,
all similar in expressive power and semantic underpinnings.  There
was a strong consensus at this meeting that more widespread use of
this class of functional languages\index{functional language} was
being hampered by the lack of a common language.  It was decided
that a committee should be formed to design such a language, providing
faster communication of new ideas, a stable foundation for real
applications development, and a vehicle through which others
would be encouraged to use functional languages.  This
document describes the result of that committee's efforts: a purely
functional programming language called \Haskell{},
\index{Haskell@\Haskell{}}
named after the logician Haskell B.~Curry\index{Curry, Haskell B.}
whose work provides the logical basis for much of ours.

\subsection*{Goals}

The committee's primary goal was to design a language that
satisfied these constraints:
\begin{enumerate}
\item It should be suitable for teaching, research, and applications,
      including building large systems.
\item It should be completely described via the publication of a formal
      syntax and semantics.
\item It should be freely available.  Anyone should be permitted to
      implement the language and distribute it to whomever they please.
\item It should be based on ideas that enjoy a wide consensus.  
\item It should reduce unnecessary diversity in functional 
      programming languages.
\end{enumerate}
The committee hopes that \Haskell{} can serve as a
basis for future research in language design.
We hope that extensions or
variants of the language may appear, incorporating experimental
features.

\subsection*{This Report}

This report is the official specification of the \Haskell{}
language and should be suitable for writing programs and building
implementations.  It is {\em not} a tutorial on programming in
\Haskell{}, so some familiarity with functional languages is assumed.
As this is the first edition of the specification, there may be some errors
and inconsistencies; beware.

\subsection*{The Next Stage}

\Haskell{} is a large and complex language, designed
for a wide spectrum of purposes.  It also introduces a major new 
technical innovation, namely using type classes to handle overloading in
a systematic way.  This innovation permeates every aspect of the language.

\Haskell{} is bound to contain infelicities and
errors of judgement.   We welcome your
comments, suggestions, and criticisms on the language or its presentation in
the report.  Together with your input and our own experience of using
the language, we plan to meet at some future time to resolve
difficulties and further stabilise the design.

A common mailing list for technical discussion of \Haskell{} can be
reached at either \mbox{\tt haskell@cs.yale.edu} or \mbox{\tt haskell@dcs.glasgow.ac.uk}.
\label{haskell-mailing-list}
\index{Haskell mailing list@\Haskell{} mailing list}
Errata sheets for this report will be posted there.
To subscribe, send a request to 
\mbox{\tt haskell-request@dcs.glasgow.ac.uk} (European residents) or
\mbox{\tt haskell-request@cs.yale.edu} (residents elsewhere).

We thought it would be helpful to identify the aspects of the language
design that seem to be most finely balanced, and hence are the
most likely candidates for change when we review the language.
The following list summarises these areas.  It will only be fully
comprehensible after you have read the report.

\paragraph*{Mutually recursive modules.}
Mutual recursion among modules is unrestricted at pre\-sent, which is
obviously desirable from the programmer's point of view, but which poses
significant challenges to the compilation system.  In particular, it is
{\em not} sufficient to start with trivial interfaces for each module and
iterate to a fixpoint, as this example shows:
\bprog
\mbox{\tt module\ F(\ f\ )\ where}\\
\mbox{\tt \ \ \ \ \ \ \ \ import\ G}\\
\mbox{\tt \ \ \ \ \ \ \ \ f\ [x]\ =\ g\ x}\\
\mbox{\tt }\\[-8pt]
\mbox{\tt module\ G(\ g\ )\ where}\\
\mbox{\tt \ \ \ \ \ \ \ \ import\ F}\\
\mbox{\tt \ \ \ \ \ \ \ \ g\ =\ f}
\eprog
If a compilation system starts off by giving \mbox{\tt F} and \mbox{\tt G} interfaces
that give the type signatures \mbox{\tt f::a} and \mbox{\tt g::b} respectively, then
compiling the two modules alternately will not reach a fixed point.
(This only happens if there is a type error, but it is obviously
undesirable behaviour.)  In general, a compiler may need to analyse a
set of mutually recursive modules as a whole, rather than separately.

\paragraph*{Generalising type classes.}  A number of restrictions are
placed on the class system in \Haskell{}.  Currently, instances
are attached to the top level type of an object and are exported
implicitly with classes and types.  A number of proposals for
generalising the class system have been discussed, among them attaching
instances to more complex types, parameterising classes over type
constructors, allowing redefinition of instances, and making instances
explicit in import and export lists.  Some of these proposals have
been implemented and are part of the available \Haskell{} systems.  As
we gain more experience with the class system we hope to improve it
in the future.

\paragraph*{Default methods.}
\index{default method}
Section~\ref{class-decls} describes how a class declaration may
include default methods for some of its operations.  We considered extending
this so that a class declaration could include default methods {\em for
operations of its superclasses}, which override the superclass's default
method.  This looks like an attractive idea, which will certainly
be considered for a future revision.

\paragraph*{Defaults for ambiguous types.}
Section~\ref{default-decls}
describes how ambiguous typings, which arise due to the type-class system,
are resolved.  Ideally, the choice made should not matter. For example,
consider the expression \mbox{\tt if\ round\ x\ >\ 0\ then\ E1\ else\ E2}.  It should
not matter whether \mbox{\tt round} returns \mbox{\tt Int} or \mbox{\tt Integer};
a bad choice could result in overflow, or a less efficient
program, but if a result is produced it will be correct.  

Our resolution rules strive only to resolve ambiguous
types where the type chosen does not ``matter'' in this sense, but we have
not been entirely successful, for example where floating point is concerned.
Further research and practical experience may suggest a better set of rules.

\paragraph*{Static semantics of \mbox{\tt let} and \mbox{\tt where} bindings.}
\label{next-stage-monomorphic}

The rules at the end of Section~\ref{pattern-bindings} comprise the
``monomorphism restriction''
\index{monomorphism restriction}
in \Haskell{}.  The
restriction solves two problems, which are summarised below, but at
the cost of restricting expressiveness.  Only experience will tell how
much of a problem this is for the programmer.

These are the two problems.  First, the expression
\mbox{\tt let\ x\ =\ factorial\ 1000\ in\ (x,x)} looks as though 
\mbox{\tt x} should only be computed once.  If \mbox{\tt x} were used
at different overloadings, however, \mbox{\tt factorial\ 1000} would be computed 
twice, once at each type.  We have found examples where the loss of efficiency
is exponential in the size of the program.  
Modest compiler optimisations can often eliminate
the problem, but we have found no simple scheme that can {\em guarantee} to do so.
The restriction solves the problem by ensuring that all uses of \mbox{\tt x} are
at the same overloading, and hence that its evaluation can be shared as usual.

Second, a rather subtle form of type 
ambiguity (Section~\ref{default-decls})
is eliminated by the restriction to non-overloaded pattern bindings.
An example is:
\bprog
\mbox{\tt readNum\ s\ r\ =\ (n*r,s')\ where\ [(n,s')]\ =\ reads\ s}
\eprog
Here \mbox{\tt n::(Num\ a,\ Text\ a)\ =>\ a}, \mbox{\tt s'::Text\ a\ =>\ String}.  If the
definition of \mbox{\tt [(n,s')]} is polymorphic, the \mbox{\tt a}'s may be resolved as
different types.

(Note: As of the version~1.1 report, the monomorphism restriction is
relaxed, provided that the programmer gives an explicit type
signature.  See Section~\ref{sect:monomorphism-restriction} for
precise details.)


\paragraph*{Overloaded constants.}
Overloaded constants (e.g.~\mbox{\tt 1}, which has type \mbox{\tt Num\ a\ =>\ a}) are
extraordinarily convenient when programming, but are the source of
several serious technical problems, including both of those mentioned
in the two preceding items.  One could eliminate overloaded
constants altogether; we considered this at length, and we are sure to
reconsider it when we review the language.

\paragraph*{Polymorphism in \mbox{\tt case} expressions.}

The type of a variable bound by a Standard~ML case-expression is monomorphic;
\index{monomorphic type variable}
we have made the same decision in \Haskell{}
(Section \ref{case-semantics}).
The question of whether such types can be made polymorphic interacts
with the restrictions on polymorphism for pattern-bound variables,
mentioned above.  For the present, we have erred on the side of conservatism,
but this decision should be reviewed.

%old:
%There is no technical reason why
%the type of such a variable should not be polymorphic; in such a case,
%the translation between \mbox{\tt let} expressions and
%\mbox{\tt case} expressions would preserve the static semantics.

%We have erred on the side of conservatism, but this
%decision will be reviewed.  If implemented, such a change would be
%upward-compatible.

\subsection*{Acknowledgements}

We heartily thank these people for their useful contributions
to this report:
Lennart Augustsson,
Richard Bird,
Stephen Blott,
Tom Blenko,
Duke Briscoe,
Chris Clack,
Guy Cousineau,
Tony Davie,
Chris Fasel,
Pat Fasel, 
Bob Hiromoto,
Nic Holt,
Simon B.~Jones, 
Stef Joosten, 
Mike Joy,
Richard Kelsey,
Siau-Cheng Khoo, 
Amir Kishon, 
John Launchbury,
Olaf Lubeck, 
Randy Michelsen, 
Rick Mohr,
Arthur Norman,
Paul Otto, 
Larne Pekowsky,
John Peterson,
Rinus Plasmeijer,  
John Robson, 
Colin Runciman, 
Lauren Smith, 
Raman Sundaresh,
Tom Thomson,
Pradeep Varma,
Tony Warnock,
Stuart Wray,
and Bonnie Yantis.
We also thank those who participated in the lively discussions
about \Haskell{} on the FP mailing list during an interim period of
the design.

%We owe a particular debt to Mar{\'\i}a Guzm{\'a}n at Yale and Will Partain
%at Glasgow, who have spent many hours working on the details
%and typography of the report.

Finally, aside from the important foundational work laid by Church,
Rosser, Curry, and others on the lambda calculus, we wish to
acknowledge the influence of many noteworthy programming languages
developed over the years.  Although it is difficult to pinpoint the
origin of many ideas, we particularly wish to acknowledge the
influence of McCarthy's Lisp \cite{mcca60} (and its modern-day
incarnation, Scheme \cite{RRRRS}); Landin's ISWIM \cite{landin66};
Backus's FP \cite{back78}; Gordon, Milner, and Wadsworth's ML
\cite{gordonetal78}; Burstall, MacQueen, and Sannella's Hope
\cite{burs80}; and Turner's series of languages culminating in
Miranda \cite{turn85}.\footnote{{\rm Miranda} is a trademark of
Research Software Ltd.} Without these forerunners \Haskell{} would
not have been possible.

\begin{flushright}
The \Haskell{} Committee\\
1 April 1990\\
(revised) 19 August 1991
\end{flushright}

% Local Variables: 
% mode: latex
% End:

